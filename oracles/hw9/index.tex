% Created 2021-04-19 Mon 21:42
% Intended LaTeX compiler: xelatex
\documentclass[12pt]{article}
\usepackage{graphicx}
\usepackage{grffile}
\usepackage{longtable}
\usepackage{wrapfig}
\usepackage{rotating}
\usepackage[normalem]{ulem}
\usepackage{amsmath}
\usepackage{textcomp}
\usepackage{amssymb}
\usepackage{capt-of}
\usepackage{hyperref}
\usepackage{minted}
\usepackage{amsmath}
\usepackage{amssymb}
\usepackage{setspace}
\usepackage{subcaption}
\usepackage{mathtools}
\usepackage{xfrac}
\usepackage[margin=1in]{geometry}
\usepackage{marginnote}
\usepackage[utf8]{inputenc}
\usepackage{color}
\usepackage{epsf}
\usepackage{tikz}
\usepackage{graphicx}
\usepackage{pslatex}
\usepackage{hyperref}

\usepackage{concmath}
\usepackage[OT1]{fontenc}

\usepackage{textgreek}
\renewcommand*{\textgreekfontmap}{%
{phv/*/*}{LGR/neohellenic/*/*}%
{*/b/n}{LGR/artemisia/b/n}%
{*/bx/n}{LGR/artemisia/bx/n}%
{*/*/n}{LGR/artemisia/m/n}%
{*/b/it}{LGR/artemisia/b/it}%
{*/bx/it}{LGR/artemisia/bx/it}%
{*/*/it}{LGR/artemisia/m/it}%
{*/b/sl}{LGR/artemisia/b/sl}%
{*/bx/sl}{LGR/artemisia/bx/sl}%
{*/*/sl}{LGR/artemisia/m/sl}%
{*/*/sc}{LGR/artemisia/m/sc}%
{*/*/sco}{LGR/artemisia/m/sco}%
}
\makeatletter
\newcommand*{\rom}[1]{\expandafter\@slowromancap\romannumeral #1@}
\makeatother
\DeclarePairedDelimiterX{\infdivx}[2]{(}{)}{%
#1\;\delimsize\|\;#2%
}
\newcommand{\infdiv}{D\infdivx}
\DeclarePairedDelimiter{\norm}{\left\lVert}{\right\rVert}
\DeclarePairedDelimiter{\ceil}{\left\lceil}{\right\rceil}
\DeclarePairedDelimiter{\floor}{\left\lfloor}{\right\rfloor}
\def\Z{\mathbb Z}
\def\R{\mathbb R}
\def\C{\mathbb C}
\def\N{\mathbb N}
\def\Q{\mathbb Q}
\def\noi{\noindent}
\onehalfspace
\usemintedstyle{bw}
\author{Sandy Urazayev\thanks{University of Kansas (ctu@ku.edu)}}
\date{109; 12021 H.E.}
\title{Homework 9 Oracle\\\medskip
\large MATH 220 Spring 2021}
\hypersetup{
 pdfauthor={Sandy Urazayev},
 pdftitle={Homework 9 Oracle},
 pdfkeywords={},
 pdfsubject={},
 pdfcreator={Emacs 28.0.50 (Org mode 9.4.5)}, 
 pdflang={English}}
\begin{document}

\maketitle
\href{./index.pdf}{[View the PDF version]​}

\section*{Section 7.5}
\label{sec:org38917ce}
\subsection*{Problem 9 (this is graded)}
\label{sec:org3e35be9}
Let us solve the following system of ODEs
\begin{align*}
        \mathbf{x}' =
        \begin{pmatrix}
                1 & -1 & 4  \\
                3 & 2  & -1 \\
                2 & 1  & -1
        \end{pmatrix} \mathbf{x}
\end{align*}

\uline{Solution}
Let \(\mathbf{A}\) be the matrix above, so let us find the characteristic
polynomial of our system
\begin{align*}
        \det (\mathbf{A} - \lambda \mathbf{I}_3) =
        \det \begin{pmatrix}
                1 - \lambda & -1          & 4            \\
                3           & 2 - \lambda & -1           \\
                2           & 1           & -1 - \lambda
        \end{pmatrix} & =
        - \lambda^3 + 2 \lambda^2 + 5\lambda -6 = 0                                                                                        \\
                                       & \implies \lambda_1 = 1 \quad \text{and} \quad \lambda_2 = -2 \quad \text{and} \quad \lambda_3 = 3
\end{align*}
Substitute the eigenvalues above to find the corresponding eigenvectors
\(\{\mathbf{v}_1, \mathbf{v}_2, \mathbf{v}_3\}\)
\begin{align*}
        \lambda = \lambda_1
        \implies
        \begin{pmatrix}
                1 - \lambda_1 & -1            & 4              \\
                3             & 2 - \lambda_1 & -1             \\
                2             & 1             & -1 - \lambda_1
        \end{pmatrix} \mathbf{v}_1
        \implies
        \mathbf{v}_1 =
        \begin{pmatrix}
                -1 \\ 4 \\ 1
        \end{pmatrix}
\end{align*}
\begin{align*}
        \lambda = \lambda_2
        \implies
        \begin{pmatrix}
                1 - \lambda_2 & -1            & 4              \\
                3             & 2 - \lambda_2 & -1             \\
                2             & 1             & -1 - \lambda_2
        \end{pmatrix} \mathbf{v}_2
        \implies
        \mathbf{v}_2 =
        \begin{pmatrix}
                -1 \\ 1 \\ 1
        \end{pmatrix}
\end{align*}
 \begin{align*}
        \lambda = \lambda_3
        \implies
        \begin{pmatrix}
                1 - \lambda_3 & -1            & 4              \\
                3             & 2 - \lambda_3 & -1             \\
                2             & 1             & -1 - \lambda_3
        \end{pmatrix} \mathbf{v}_3
        \implies
        \mathbf{v}_3 =
        \begin{pmatrix}
                1 \\ 2 \\ 1
        \end{pmatrix}
\end{align*}
Therefore the final solution has the form
\begin{align*}
        \mathbf{x}(t) & = C_1 e^{\lambda_1 t} \mathbf{v_1} + C_2 e^{\lambda_2 t} \mathbf{v_2} + C_3 e^{\lambda_3 t} \mathbf{v_3}\\
                   & = C_1 e^{t} \begin{pmatrix}
                -1 \\ 4 \\ 1
        \end{pmatrix}
        + C_2 e^{-2t} \begin{pmatrix}
                -1 \\ 1 \\ 1
        \end{pmatrix}
        + C_3 e^{3t} \begin{pmatrix}
                1 \\ 2 \\ 1
        \end{pmatrix}
\end{align*}
where \(C_1\) and \(C_2\) and \(C_3\) are arbitrary constants.
\subsection*{Problem 10}
\label{sec:org0826278}
The solution strategy is the same as \textbf{Problem 9}. We find that eigenvalues are
\(\lambda_1 = 4\) and \(\lambda_2 = 2\) with their corresponding eigenvectors
\(\mathbf{v}_1 = \begin{pmatrix}1 \\ 1 \end{pmatrix}\) and
\(\mathbf{v}_2 = \begin{pmatrix}1 \\ 3 \end{pmatrix}\), such that the solution
is
\begin{align*}
        \mathbf{x}(t) = C_1 e^{4t} \begin{pmatrix}1 \\ 1 \end{pmatrix} +
        C_2 e^{2t} \begin{pmatrix}1 \\ 3 \end{pmatrix}
\end{align*}
By knowing the initial value
\(\mathbf{x}(0) = \begin{pmatrix} 2 \\ -1 \end{pmatrix}\), we solve the system
\begin{align*}
        \begin{cases}
                C_1 + C_2 = 2 \\
                C_1 + 3 C_2 = -1
        \end{cases}
        \implies
        \begin{cases}
                C_1 = \frac{7}{2} \\
                C_2 = -\frac{3}{2}
        \end{cases}
\end{align*}
Finally, the solution to the initial value problem is
\begin{align*}
        \mathbf{x}(t) = \frac{7}{2} e^{4t} \begin{pmatrix}1 \\ 1 \end{pmatrix} +
        - \frac{3}{2} e^{2t} \begin{pmatrix}1 \\ 3 \end{pmatrix}
\end{align*}
\subsection*{Problem 15 (this is graded)}
\label{sec:org2a1ebc4}
Recall the eigenvalues and their eigenvectors we found in \textbf{Problem
10}. Applying \textbf{Problem 13}, the solution is
\begin{align*}
        \mathbf{x}(t) = C_1 t^4 \begin{pmatrix} 1\\ 1\end{pmatrix}
        + C_2 t^2 \begin{pmatrix} 1\\ 3\end{pmatrix}
\end{align*}
\subsection*{Problem 16}
\label{sec:org05c00a8}
Similarly to all problems above, we find eigenvalues
\(\lambda_1 = 0\) and \(\lambda_2 = -2\), where their respective eigenvectors are
\(\mathbf{v}_1 = \begin{pmatrix} 3 \\ 4\end{pmatrix}\) and
\(\mathbf{v}_2 = \begin{pmatrix} 1 \\ 2\end{pmatrix}\). Applying \textbf{Problem 13}, we
have the solution
\begin{align*}
        \mathbf{x}(t) = C_1 \begin{pmatrix} 3\\ 4\end{pmatrix} +
        C_2 t^{-2} \begin{pmatrix} 1\\ 2\end{pmatrix}
\end{align*}
\section*{Section 7.6}
\label{sec:orgf6210bf}
\subsection*{Problem 9 and 10}
\label{sec:orgf005e04}
Solutions are omitted for the sake of expediting the grading before Midterm
II. 
\end{document}