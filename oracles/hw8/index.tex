% Created 2021-04-19 Mon 20:17
% Intended LaTeX compiler: xelatex
\documentclass[12pt]{article}
\usepackage{graphicx}
\usepackage{grffile}
\usepackage{longtable}
\usepackage{wrapfig}
\usepackage{rotating}
\usepackage[normalem]{ulem}
\usepackage{amsmath}
\usepackage{textcomp}
\usepackage{amssymb}
\usepackage{capt-of}
\usepackage{hyperref}
\usepackage{minted}
\usepackage{amsmath}
\usepackage{amssymb}
\usepackage{setspace}
\usepackage{subcaption}
\usepackage{mathtools}
\usepackage{xfrac}
\usepackage[margin=1in]{geometry}
\usepackage[utf8]{inputenc}
\usepackage{color}
\usepackage{epsf}
\usepackage{tikz}
\usepackage{graphicx}
\usepackage{pslatex}
\usepackage{hyperref}

\usepackage{concmath}
\usepackage[OT1]{fontenc}

\usepackage{textgreek}
\renewcommand*{\textgreekfontmap}{%
{phv/*/*}{LGR/neohellenic/*/*}%
{*/b/n}{LGR/artemisia/b/n}%
{*/bx/n}{LGR/artemisia/bx/n}%
{*/*/n}{LGR/artemisia/m/n}%
{*/b/it}{LGR/artemisia/b/it}%
{*/bx/it}{LGR/artemisia/bx/it}%
{*/*/it}{LGR/artemisia/m/it}%
{*/b/sl}{LGR/artemisia/b/sl}%
{*/bx/sl}{LGR/artemisia/bx/sl}%
{*/*/sl}{LGR/artemisia/m/sl}%
{*/*/sc}{LGR/artemisia/m/sc}%
{*/*/sco}{LGR/artemisia/m/sco}%
}
\makeatletter
\newcommand*{\rom}[1]{\expandafter\@slowromancap\romannumeral #1@}
\makeatother
\DeclarePairedDelimiterX{\infdivx}[2]{(}{)}{%
#1\;\delimsize\|\;#2%
}
\newcommand{\infdiv}{D\infdivx}
\DeclarePairedDelimiter{\norm}{\left\lVert}{\right\rVert}
\DeclarePairedDelimiter{\ceil}{\left\lceil}{\right\rceil}
\DeclarePairedDelimiter{\floor}{\left\lfloor}{\right\rfloor}
\def\Z{\mathbb Z}
\def\R{\mathbb R}
\def\C{\mathbb C}
\def\N{\mathbb N}
\def\Q{\mathbb Q}
\def\noi{\noindent}
\onehalfspace
\usemintedstyle{bw}
\author{Sandy Urazayev\thanks{University of Kansas (ctu@ku.edu)}}
\date{107; 12021 H.E.}
\title{Homework 8 Oracle\\\medskip
\large MATH 220 Spring 2021}
\hypersetup{
 pdfauthor={Sandy Urazayev},
 pdftitle={Homework 8 Oracle},
 pdfkeywords={},
 pdfsubject={},
 pdfcreator={Emacs 28.0.50 (Org mode 9.4.5)}, 
 pdflang={English}}
\begin{document}

\maketitle
\href{./index.pdf}{[View the PDF version]​}

\section*{Section 7.3}
\label{sec:org09b0b98}
\subsection*{Problem 7}
\label{sec:orgd488459}
   The three vectors are linearly dependent if there exists a nontrivial solution to
$$
c_{1} \mathbf{x}^{(1)}+c_{2} \mathbf{x}^{(2)}+c_{3} \mathbf{x}^{(3)}=\mathbf{0}
$$
for \(c_{1}, c_{2}\), and \(c_{3}\). Rewrite this equation.

\begin{equation*}
	\begin{array}{l}
		c_{1}\left(\begin{array}{l}
				2 \\
				1 \\

			\end{array}\right)+c_{2}\left(\begin{array}{l}
				0 \\
				1 \\

			\end{array}\right)+c_{3}\left(\begin{array}{r}
				-1 \\
				2  \\

			\end{array}\right)=\left(\begin{array}{l}
				0 \\
				0 \\

			\end{array}\right) \\
		\left(\begin{array}{llr}
				2 & 0 & -1 \\
				1 & 1 & 2  \\
				0 & 0 & 0
			\end{array}\right)\left(\begin{array}{l}
				c_{1} \\
				c_{2} \\
				c_{3}
			\end{array}\right)=\left(\begin{array}{l}
				0 \\
				0 \\

			\end{array}\right)
	\end{array}
\end{equation*}

Calculate the determinant of the coefficient matrix.
$$
	\operatorname{det}\left(\begin{array}{rrr}
			2 & 0 & -1 \\
			1 & 1 & 2  \\
			0 & 0 & 0
		\end{array}\right)=0\left(\begin{array}{rr}
			0 & -1 \\
			1 & 2
		\end{array}\right)-0\left(\begin{array}{rr}
			2 & -1 \\
			1 & 2
		\end{array}\right)+0\left(\begin{array}{ll}
			2 & 0 \\
			1 & 1
		\end{array}\right)=0
$$

Since it's zero, there are infinitely many solutions for \(c_{1}, c_{2}\), and \(c_{3}\).

\begin{array}{r}
2 c_{1}-c_{3}=0 \\
c_{1}+c_{2}+2 c_{3}=0
\end{array}

Solve this first equation for \(c_{3}\)
$$
c_{3}=2 c_{1}
$$

and plug it into the second one.

$$
c_{1}+c_{2}+2\left(2 c_{1}\right)=0
$$

Solve for \(c_{2}\)
$$
c_{2}=-5 c_{1}
$$
In terms of the free variable \(c_{1}\), the solution to the system of equations is
$$
\left\{c_{1},-5 c_{1}, 2 c_{1}\right\}
$$
For example, choose \(c_{1}=1\). Then
$$
\mathbf{x}^{(1)}-5 \mathbf{x}^{(2)}+2 \mathbf{x}^{(3)}=\mathbf{0}
$$
Therefore, the three given vectors are linearly dependent.

\subsection*{Problem 16 [FOR GRADE]}
\label{sec:org6ac19d1}
   The aim is to solve the eigenvalue problem,
$$
\mathbf{A} \mathbf{x}=\lambda \mathbf{x}
$$
where \(\mathbf{A}\) is the given matrix. Bring \(\lambda \mathbf{x}\) to the left side and combine the terms.
$$
(\mathbf{A}-\lambda \mathbf{I}) \mathbf{x}=\mathbf{0}
$$
The eigenvalues satisfy
$$
\operatorname{det}(\mathbf{A}-\lambda \mathbf{I})=0
$$
Evaluate the determinant and solve for \(\lambda\).
\begin{equation*}
	\begin{array}{c}
		\operatorname{det}\left(\begin{array}{cc}
				-2-\lambda & 1          \\
				1          & -2-\lambda
			\end{array}\right)=0 \\
		(-2-\lambda)(-2-\lambda)-1=0                               \\
		\lambda^{2}+4 \lambda+3=0                                  \\
		(\lambda+3)(\lambda+1)=0                                   \\
		\lambda=\{-3,-1\}
	\end{array}
\end{equation*}
Therefore, the eigenvalues are
\begin{equation*}
\lambda_{1}=-3 \text { and } \quad \lambda_{2}=-1
\end{equation*}
Substitute \(\lambda_{1}\) and \(\lambda_{2}\) back into equation (1) to determine
the corresponding eigenvectors, \(\mathbf{x}_{1}\) and \(\mathbf{x}_{2}\)
\begin{equation*}
	\begin{array}{r}
		\left(\mathbf{A}-\lambda_{1} \mathbf{I}\right) \mathbf{x}_{1}=\mathbf{0}                                            & \left(\mathbf{A}-\lambda_{2} \mathbf{I}\right) \mathbf{x}_{2}=\mathbf{0}                                            \\
		\left(\begin{array}{ll}
				1 & 1 \\
				1 & 1
			\end{array}\right)\left(\begin{array}{l}
				x_{1} \\
				x_{2}
			\end{array}\right)=\left(\begin{array}{l}
				0 \\

			\end{array}\right) & \left(\begin{array}{cc}
				-1 & 1  \\
				1  & -1
			\end{array}\right)\left(\begin{array}{l}
				x_{1} \\
				x_{2}
			\end{array}\right)=\left(\begin{array}{l}
				0 \\

			\end{array}\right) \\
		x_{1}+x_{2}=0                                                                                                       & -x_{1}+x_{2}=0                                                                                                      \\
		\left.x_{1}+x_{2}=0\right\}                                                                                         & x_{1}-x_{2}=0                                                                                                       \\
		x_{2}=-x_{1}                                                                                                        & x_{2}=x_{1}                                                                                                         \\
		\mathbf{x}_{1}=\left(\begin{array}{c}
				x_{1} \\
				-x_{1}
			\end{array}\right)=x_{1}\left(\begin{array}{c}
				1 \\
				-1
			\end{array}\right)                   & \mathbf{x}_{2}=\left(\begin{array}{l}
				x_{1} \\
				x_{1}
			\end{array}\right)=x_{1}\left(\begin{array}{l}
				1 \\
				1
			\end{array}\right)
	\end{array}
\end{equation*}
Note that \(x_{1}\) is a free variable, or arbitrary constant.
\subsection*{Problem 17}
\label{sec:org76d0975}
   The aim is to solve the eigenvalue problem,
$$
\mathbf{A x}=\lambda \mathbf{x}
$$
where \(\mathbf{A}\) is the given matrix. Bring \(\lambda \mathbf{x}\) to the left side and combine the terms.
$$
(\mathbf{A}-\lambda \mathbf{I}) \mathbf{x}=\mathbf{0}
$$
The eigenvalues satisfy
$$
\operatorname{det}(\mathbf{A}-\lambda \mathbf{I})=0
$$
Evaluate the determinant and solve for \(\lambda\).
\begin{equation*}
  \begin{array}{c}
    \operatorname{det}\left(\begin{array}{cc}
                              1-\lambda & \sqrt{3} \\
                              \sqrt{3} & -1-\lambda
                            \end{array}\right)=0 \\
    (1-\lambda)(-1-\lambda)-3=0 \\
    \lambda^{2}-4=0 \\
    (\lambda+2)(\lambda-2)=0 \\
    \lambda=\{-2,2\}
  \end{array}
\end{equation*}
Therefore, the eigenvalues are
\begin{equation*}
\lambda_{1}=-2 \text { and } \quad \lambda_{2}=2
\end{equation*}
Substitute \(\lambda_{1}\) and \(\lambda_{2}\) back into equation (1) to determine
the corresponding eigenvectors, \(\mathbf{x}_{1}\) and \(\mathbf{x}_{2}\)

\begin{equation*}
	\begin{array}{c}
		\left(\mathbf{A}-\lambda_{1} \mathbf{I}\right) \mathbf{x}_{1}=\mathbf{0} \\
		\left(\begin{array}{cc}
				3        & \sqrt{3} \\
				\sqrt{3} & 1
			\end{array}\right)\left(\begin{array}{l}
				x_{1} \\
				x_{2}
			\end{array}\right)=\left(\begin{array}{l}
				0 \\

			\end{array}\right)
	\end{array}
\end{equation*}

\begin{equation*}
	\left.\begin{array}{r}
		3 x_{1}+\sqrt{3} x_{2}=0 \\
		\sqrt{3} x_{1}+x_{2}=0
	\end{array}\right\}
\end{equation*}

\begin{equation*}
	\begin{array}{r}
		x_{2}=-\sqrt{3} x_{1} \\
		\mathbf{x}_{1}=\left(\begin{array}{c}
				x_{1} \\
				-\sqrt{3} x_{1}
			\end{array}\right)=x_{1}\left(\begin{array}{c}
				1 \\
				-\sqrt{3}
			\end{array}\right)
	\end{array}
\end{equation*}

and

\begin{equation*}
	\begin{array}{c}
		\left(\mathbf{A}-\lambda_{2} \mathbf{I}\right) \mathbf{x}_{2}=\mathbf{0} \\
		\left(\begin{array}{ll}
				-1       & \sqrt{3} \\
				\sqrt{3} & -3
			\end{array}\right)\left(\begin{array}{l}
				x_{1} \\
				x_{2}
			\end{array}\right)=\left(\begin{array}{l}
				0 \\

			\end{array}\right)
	\end{array}
\end{equation*}

\begin{equation*}
	\left.\begin{array}{r}
		-x_{1}+\sqrt{3} x_{2}=0 \\
		\sqrt{3} x_{1}-3 x_{2}=0
	\end{array}\right\}
\end{equation*}

\begin{equation*}
	\begin{array}{c}
		x_{2}=\frac{1}{\sqrt{3}} x_{1} \\
		\mathbf{x}_{2}=\left(\begin{array}{c}
				x_{1} \\
				\frac{1}{\sqrt{3}} x_{1}
			\end{array}\right)=x_{1}\left(\begin{array}{c}
				1 \\
				\frac{1}{\sqrt{3}}
			\end{array}\right)
	\end{array}
\end{equation*}

Note that \(x_{1}\) is a free variable, or arbitrary constant.
\subsection*{Problem 18}
\label{sec:org6b68ad9}
   The aim is to solve the eigenvalue problem,
$$
\mathbf{A} \mathbf{x}=\lambda \mathbf{x}
$$
where \(\mathbf{A}\) is the given matrix. Bring \(\lambda \mathbf{x}\) to the left side and combine the terms.
$$
(\mathbf{A}-\lambda \mathbf{I}) \mathbf{x}=\mathbf{0}
$$
The eigenvalues satisfy
$$
\operatorname{det}(\mathbf{A}-\lambda \mathbf{I})=0
$$
Evaluate the determinant and solve for \(\lambda\).
\begin{equation*}
	\begin{array}{c}
		\operatorname{det}\left(\begin{array}{ccc}
				1-\lambda & 0         & 0         \\
				2         & 1-\lambda & -2        \\
				3         & 2         & 1-\lambda
			\end{array}\right)=0     \\
		(1-\lambda)\left|\begin{array}{cc}
			1-\lambda & -2        \\
			2         & 1-\lambda
		\end{array}\right|=0            \\
		(1-\lambda)[(1-\lambda)(1-\lambda)+4]=0                        \\
		1-\lambda=0 \quad \text { or } \quad \lambda^{2}-2 \lambda+5=0 \\
		\quad 5-7 \lambda+3 \lambda^{2}-\lambda^{3}=0                  \\
		(1-\lambda)\left(\lambda^{2}-2 \lambda+5\right)=0              \\
		\lambda=1 \quad \text { or } \quad \lambda=\frac{2 \pm \sqrt{4-20}}{2}=1 \pm 2 i
	\end{array}
\end{equation*}

Therefore, the eigenvalues are

\begin{align*}
\begin{array}{|l|l|l|l|}
\hline \lambda_{1}=1 & \lambda_{2}=1-2 i & \lambda_{3}=1+2 i
\end{array}
\end{align*}

Substitute \(\lambda_{1}\) back into equation (1) to determine the corresponding
eigenvector \(\mathbf{x}_{1}\). 

\begin{equation*}
	\begin{array}{c}
		\left(\mathbf{A}-\lambda_{1} \mathbf{I}\right) \mathbf{x}_{1}=\mathbf{0}                                            \\
		\left(\begin{array}{ccc}
				1-(1) & 0     & 0     \\
				2     & 1-(1) & -2    \\
				3     & 2     & 1-(1)
			\end{array}\right)\left(\begin{array}{l}
				x_{1} \\
				x_{2} \\
				x_{3}
			\end{array}\right)=\left(\begin{array}{c}
				0 \\
				0 \\

			\end{array}\right) \\
		\left(\begin{array}{ccc}
				0 & 0 & 0  \\
				2 & 0 & -2 \\
				3 & 2 & 0
			\end{array}\right)\left(\begin{array}{l}
				x_{1} \\
				x_{2} \\
				x_{3}
			\end{array}\right)=\left(\begin{array}{l}
				0 \\
				0 \\

			\end{array}\right)
	\end{array}
\end{equation*}

Write the implied system of equations.

\begin{equation*}
	\left.\begin{array}{l}
		2 x_{1}-2 x_{3}=0 \\
		3 x_{1}+2 x_{2}=0
	\end{array}\right\}
\end{equation*}

Solve for \(x_{2}\) and \(x_{3}\) in terms of the free variable \(x_{1}\).

\begin{array}{l}
x_{3}=x_{1} \\
x_{2}=-\frac{3}{2} x_{1}
\end{array}

This means

\begin{equation*}
\mathbf{x}_{1}=\left(\begin{array}{l}
x_{1} \\
x_{2} \\
x_{3}
\end{array}\right)=\left(\begin{array}{c}
x_{1} \\
-\frac{3}{2} x_{1} \\
x_{1}
\end{array}\right)=x_{1}\left(\begin{array}{c}
1 \\
-\frac{3}{2} \\
1
\end{array}\right)
\end{equation*}

Since \(x_1\) is arbitrary, the eigenvector can be multiplied by 2 to get rid of the
fraction. 

\begin{equation*}
	\mathbf{x}_{1}=x_{1}^{\prime}\left(\begin{array}{c}
			2  \\
			-3 \\
			2
		\end{array}\right)
\end{equation*}

\subsection*{Problem 20 [FOR GRADE]}
\label{sec:orga3e8ee3}
   The aim is to solve the eigenvalue problem,
$$
\mathbf{A} \mathbf{x}=\lambda \mathbf{x}
$$
where \(\mathbf{A}\) is the given matrix. Bring \(\lambda \mathbf{x}\) to the left
side and combine the terms. 
$$
(\mathbf{A}-\lambda \mathbf{I}) \mathbf{x}=\mathbf{0}
$$
The eigenvalues satisfy
$$
\operatorname{det}(\mathbf{A}-\lambda \mathbf{I})=0
$$
Evaluate the determinant and solve for \(\lambda\).

\begin{equation*}
	\begin{array}{c}
		\operatorname{det}\left(\begin{array}{ccc}
				11 / 9-\lambda & -2 / 9        & 8 / 9         \\
				-2 / 9         & 2 / 9-\lambda & 10 / 9        \\
				8 / 9          & 10 / 9        & 5 / 9-\lambda
			\end{array}\right)=0                                                                                            \\
		(11 / 9-\lambda)[(2 / 9-\lambda)(5 / 9-\lambda)-100 / 81]+(2 / 9)[(-2 / 9)(5 / 9-\lambda)-80 / 81]                                                    \\
		\quad+(8 / 9)[-20 / 81-(8 / 9)(2 / 9-\lambda)]=0                                                                                                      \\
		(11 / 9-\lambda)\left|\begin{array}{cc}
			2 / 9-\lambda & 10 / 9        \\
			10 / 9        & 5 / 9-\lambda
		\end{array}\right|-(-2 / 9)\left|\begin{array}{cc}
			-2 / 9 & 10 / 9        \\
			8 / 9  & 5 / 9-\lambda
		\end{array}\right|+(8 / 9)\left|\begin{array}{cc}
			-2 / 9 & 2 / 9-\lambda \\
			8 / 9  & 10 / 9
		\end{array}\right|=0 \\
		-2+\lambda+2 \lambda^{2}-\lambda^{3}=0                                                                                                                \\
		(\lambda+1)(\lambda-1)(2-\lambda)=0
	\end{array}
\end{equation*}

Therefore, the eigenvalues are

\begin{equation*}
	\begin{array}{|l|l|l|}
		\hline \lambda_{1}=1 & \lambda_{2}=2 & \lambda_{3}=-1
	\end{array}
\end{equation*}

Substitute \(\lambda_{1}\) back into equation (1) to determine the corresponding
eigenvector \(\mathbf{x}_{1}\). 

\begin{equation*}
	\begin{array}{c}
		\left(\mathbf{A}-\lambda_{1} \mathbf{I}\right) \mathbf{x}_{1}=\mathbf{0}                                            \\
		\left(\begin{array}{ccc}
				11 / 9-(1) & -2 / 9    & 8 / 9     \\
				-2 / 9     & 2 / 9-(1) & 10 / 9    \\
				8 / 9      & 10 / 9    & 5 / 9-(1)
			\end{array}\right)\left(\begin{array}{l}
				x_{1} \\
				x_{2} \\
				x_{3}
			\end{array}\right)=\left(\begin{array}{c}
				0 \\
				0 \\

			\end{array}\right) \\
		\left(\begin{array}{ccc}
				2 / 9  & -2 / 9 & 8 / 9  \\
				-2 / 9 & -7 / 9 & 10 / 9 \\
				8 / 9  & 10 / 9 & -4 / 9
			\end{array}\right)\left(\begin{array}{l}
				x_{1} \\
				x_{2} \\
				x_{3}
			\end{array}\right)=\left(\begin{array}{c}
				0 \\
				0 \\

			\end{array}\right)
	\end{array}
\end{equation*}

Write the augmented matrix.
$$
\left(\begin{array}{rrr|r}
2 / 9 & -2 / 9 & 8 / 9 & 0 \\
-2 / 9 & -7 / 9 & 10 / 9 & 0 \\
8 / 9 & 10 / 9 & -4 / 9 & 0
\end{array}\right)
$$
Multiply each row by \(9\)
$$
\left(\begin{array}{rrr|r}
2 & -2 & 8 & 0 \\
-2 & -7 & 10 & 0 \\
8 & 10 & -4 & 0
\end{array}\right)
$$
Multiply the first row by \(-4\) and add it to the third row.
$$
\left(\begin{array}{rrr|r}
2 & -2 & 8 & 0 \\
-2 & -7 & 10 & 0 \\
0 & 18 & -36 & 0
\end{array}\right)
$$
Add the first row to the second row.
$$
\left(\begin{array}{rrr|r}
2 & -2 & 8 & 0 \\
0 & -9 & 18 & 0 \\
0 & 18 & -36 & 0
\end{array}\right)
$$

Write the implied system of equations and solve for \(x_{1}\) and \(x_{2}\) in terms
of the free variable \(x_{3}\) 
$$
\left.\begin{array}{r}
2 x_{1}-2 x_{2}+8 x_{3}=0 \\
-9 x_{2}+18 x_{3}=0 \\
18 x_{2}-36 x_{3}=0
\end{array}\right\} \quad \rightarrow \quad \begin{array}{l}
x_{1}=-2 x_{3} \\
x_{2}=2 x_{3}
\end{array}
$$
This means
$$
\mathbf{x}_{1}=\left(\begin{array}{l}
x_{1} \\
x_{2} \\
x_{3}
\end{array}\right)=\left(\begin{array}{c}
-2 x_{3} \\
2 x_{3} \\
x_{3}
\end{array}\right)
$$
Therefore,
$$
\mathbf{x}_{1}=x_{3}\left(\begin{array}{c}
-2 \\
2 \\
1
\end{array}\right)
$$
Substitute \(\lambda_{2}\) back into equation (1) to determine the corresponding eigenvector \(\mathbf{x}_{2}\).

\begin{equation*}
	\begin{array}{c}
		\left(\mathbf{A}-\lambda_{2} \mathbf{I}\right) \mathbf{x}_{2}=\mathbf{0}                                            \\
		\left(\begin{array}{ccc}
				11 / 9-(2) & -2 / 9    & 8 / 9     \\
				-2 / 9     & 2 / 9-(2) & 10 / 9    \\
				8 / 9      & 10 / 9    & 5 / 9-(2)
			\end{array}\right)\left(\begin{array}{l}
				x_{1} \\
				x_{2} \\
				x_{3}
			\end{array}\right)=\left(\begin{array}{c}
				0 \\
				0 \\

			\end{array}\right) \\
		\left(\begin{array}{ccc}
				-7 / 9 & -2 / 9  & 8 / 9   \\
				-2 / 9 & -16 / 9 & 10 / 9  \\
				8 / 9  & 10 / 9  & -13 / 9
			\end{array}\right)\left(\begin{array}{l}
				x_{1} \\
				x_{2} \\
				x_{3}
			\end{array}\right)=\left(\begin{array}{l}
				0 \\
				0 \\

			\end{array}\right)
	\end{array}
\end{equation*}

Write the augmented matrix.
$$
\left(\begin{array}{rrr|r}
-7 / 9 & -2 / 9 & 8 / 9 & 0 \\
-2 / 9 & -16 / 9 & 10 / 9 & 0 \\
8 / 9 & 10 / 9 & -13 / 9 & 0
\end{array}\right)
$$
Multiply each row by \(9\)
$$
\left(\begin{array}{rrr|r}
-7 & -2 & 8 & 0 \\
-2 & -16 & 10 & 0 \\
8 & 10 & -13 & 0
\end{array}\right)
$$
Multiply the second row by 4 and add it to the third row.
$$
\left(\begin{array}{rrr|r}
-7 & -2 & 8 & 0 \\
-2 & -16 & 10 & 0 \\
0 & -54 & 27 & 0
\end{array}\right)
$$
Multiply the first row by \(-8\) and add it to the second row.
$$
\left(\begin{array}{rrr|r}
-7 & -2 & 8 & 0 \\
54 & 0 & -54 & 0 \\
0 & -54 & 27 & 0
\end{array}\right)
$$

Write the implied system of equations and
solve for \(x_1\) and \(x_2\) in terms of the free variable \(x_3\)

\begin{equation*}
\left.\begin{array}{r}
-7 x_{1}-2 x_{2}+8 x_{3}=0 \\
54 x_{1}-54 x_{3}=0 \\
-54 x_{2}+27 x_{3}=0
\end{array}\right\} \quad \rightarrow \quad \begin{aligned}
& x_{1}=x_{3} \\
& x_{2}=\frac{1}{2} x_{3}
\end{aligned}
\end{equation*}

This means

\begin{equation*}
	\mathbf{x}_{2}=\left(\begin{array}{l}
			x_{1} \\
			x_{2} \\
			x_{3}
		\end{array}\right)=\left(\begin{array}{c}
			x_{3}             \\
			\frac{1}{2} x_{3} \\
			x_{3}
		\end{array}\right)=x_{3}\left(\begin{array}{c}
			1           \\
			\frac{1}{2} \\
			1
		\end{array}\right)
\end{equation*}

Since \(x_{3}\) is arbitrary, the eigenvector can be multiplied by 2 to get rid of
the fraction. 
\(\mathbf{x}_{2}=x_{3}^{\prime}\left(\begin{array}{l}2 \\ 1 \\ 2\end{array}\right)\)

Substitute \(\lambda_{3}\) back into equation (1) to determine the corresponding
eigenvector \(\mathbf{x}_{3}\).

\begin{equation*}
	\begin{array}{c}
		\left(\mathbf{A}-\lambda_{3} \mathbf{I}\right) \mathbf{x}_{3}=\mathbf{0}                                            \\
		\left(\begin{array}{ccc}
				11 / 9-(-1) & -2 / 9     & 8 / 9      \\
				-2 / 9      & 2 / 9-(-1) & 10 / 9     \\
				8 / 9       & 10 / 9     & 5 / 9-(-1)
			\end{array}\right)\left(\begin{array}{l}
				x_{1} \\
				x_{2} \\
				x_{3}
			\end{array}\right)=\left(\begin{array}{l}
				0 \\
				0 \\

			\end{array}\right) \\
		\left(\begin{array}{ccc}
				20 / 9 & -2 / 9 & 8 / 9  \\
				-2 / 9 & 11 / 9 & 10 / 9 \\
				8 / 9  & 10 / 9 & 14 / 9
			\end{array}\right)\left(\begin{array}{l}
				x_{1} \\
				x_{2} \\
				x_{3}
			\end{array}\right)=\left(\begin{array}{c}
				0 \\
				0 \\

			\end{array}\right)
	\end{array}
\end{equation*}

Write the augmented matrix.
$$
\left(\begin{array}{rrr|r}
20 / 9 & -2 / 9 & 8 / 9 & 0 \\
-2 / 9 & 11 / 9 & 10 / 9 & 0 \\
8 / 9 & 10 / 9 & 14 / 9 & 0
\end{array}\right)
$$
Multiply each row by \(9\)
$$
\left(\begin{array}{rrr|r}
20 & -2 & 8 & 0 \\
-2 & 11 & 10 & 0 \\
8 & 10 & 14 & 0
\end{array}\right)
$$
Multiply the second row by 4 and add it to the third row.
$$
\left(\begin{array}{rrr|r}
20 & -2 & 8 & 0 \\
-2 & 11 & 10 & 0 \\
0 & 54 & 54 & 0
\end{array}\right)
$$
Multiply the second row by 10 and add it to the first row.
$$
\left(\begin{array}{rrr|r}
0 & 108 & 108 & 0 \\
-2 & 11 & 10 & 0 \\
0 & 54 & 54 & 0
\end{array}\right)
$$

Write the implied system of equations and solve for \(x_{1}\) and \(x_{3}\) in terms
of the free variable \(x_{2}\) 
$$
\left.\begin{array}{r}
108 x_{2}+108 x_{3}=0 \\
-2 x_{1}+11 x_{2}+10 x_{3}=0 \\
54 x_{2}+54 x_{3}=0
\end{array}\right\} \quad \rightarrow \quad \begin{array}{l}
x_{1}=\frac{1}{2} x_{2} \\
x_{3}=-x_{2}
\end{array}
$$
This means
$$
\mathbf{x}_{3}=\left(\begin{array}{l}
x_{1} \\
x_{2} \\
x_{3}
\end{array}\right)=\left(\begin{array}{c}
\frac{1}{2} x_{2} \\
x_{2} \\
-x_{2}
\end{array}\right)=x_{2}\left(\begin{array}{c}
\frac{1}{2} \\
1 \\
-1
\end{array}\right)
$$

Since \(x_{2}\) is arbitrary, the eigenvector can be multiplied by 2 to get rid of
the fraction. 
$$
\mathbf{x}_{3}=x_{2}^{\prime}\left(\begin{array}{c}
1 \\
2 \\
-2
\end{array}\right)
$$

\section*{Section 7.4}
\label{sec:orga60514f}
\subsection*{Problem 5(a)}
\label{sec:org9c75e7d}
\begin{align*}
        t \begin{pmatrix}
                1 \\
                1
        \end{pmatrix} =
        \begin{pmatrix}
                2 & -1 \\
                3 & -2
        \end{pmatrix}
        \begin{pmatrix}
                t \\
                t
        \end{pmatrix}
        = \begin{pmatrix}
                2t-t \\
                3t-2t
        \end{pmatrix} =
        \begin{pmatrix}
                t \\
                t
        \end{pmatrix}
\end{align*}

\begin{align*}
        t \begin{pmatrix}
                -t^{-2} \\
                -3t^{-2}
        \end{pmatrix} =
        \begin{pmatrix}
                2 & -1 \\
                3 & -2
        \end{pmatrix}
        \begin{pmatrix}
                t^{-1} \\
                3t^{-1}
        \end{pmatrix}
        = \begin{pmatrix}
                2t^{-1}-3t^{-1} \\
                3t^{-1}-6t^{-1}
        \end{pmatrix} =
        \begin{pmatrix}
                -t^{-1} \\
                -3t^{-1}
        \end{pmatrix}
\end{align*}
\subsection*{Problem 6(a) [FOR GRADE]}
\label{sec:org8df20a7}
\begin{align*}
        t \begin{pmatrix}
                -t^{-2} \\
                -2t^{-2}
        \end{pmatrix} =
        \begin{pmatrix}
                3 & -2 \\
                2 & -2
        \end{pmatrix}
        \begin{pmatrix}
                t^{-1} \\
                2t^{-1}
        \end{pmatrix}
        = \begin{pmatrix}
                3t^{-1}-4t^{-1} \\
                2t^{-1}-4t^{-1}
        \end{pmatrix} =
        \begin{pmatrix}
                -t^{-1} \\
                -2t^{-1}
        \end{pmatrix}
\end{align*}

   \begin{align*}
        t \begin{pmatrix}
                4t \\
                2t
        \end{pmatrix} =
        \begin{pmatrix}
                3 & -2 \\
                2 & -2
        \end{pmatrix}
        \begin{pmatrix}
                2t^2 \\
                t^2
        \end{pmatrix}
        = \begin{pmatrix}
          6t^2 - 2t^2\\
          4t^2 - 2t^2
        \end{pmatrix} =
        \begin{pmatrix}
          4t^2\\
          2t^2
        \end{pmatrix}
\end{align*}
\end{document}