% Created 2021-04-21 Wed 20:59
% Intended LaTeX compiler: xelatex
\documentclass[12pt]{article}
\usepackage{graphicx}
\usepackage{grffile}
\usepackage{longtable}
\usepackage{wrapfig}
\usepackage{rotating}
\usepackage[normalem]{ulem}
\usepackage{amsmath}
\usepackage{textcomp}
\usepackage{amssymb}
\usepackage{capt-of}
\usepackage{hyperref}
\usepackage{minted}
\usepackage{amsmath}
\usepackage{amssymb}
\usepackage{setspace}
\usepackage{subcaption}
\usepackage{mathtools}
\usepackage{xfrac}
\usepackage[margin=1in]{geometry}
\usepackage{marginnote}
\usepackage[utf8]{inputenc}
\usepackage{color}
\usepackage{epsf}
\usepackage{tikz}
\usepackage{graphicx}
\usepackage{pslatex}
\usepackage{hyperref}

\usepackage{beton}
\usepackage{euler}
\usepackage[OT1]{fontenc}

\usepackage{textgreek}
\renewcommand*{\textgreekfontmap}{%
{phv/*/*}{LGR/neohellenic/*/*}%
{*/b/n}{LGR/artemisia/b/n}%
{*/bx/n}{LGR/artemisia/bx/n}%
{*/*/n}{LGR/artemisia/m/n}%
{*/b/it}{LGR/artemisia/b/it}%
{*/bx/it}{LGR/artemisia/bx/it}%
{*/*/it}{LGR/artemisia/m/it}%
{*/b/sl}{LGR/artemisia/b/sl}%
{*/bx/sl}{LGR/artemisia/bx/sl}%
{*/*/sl}{LGR/artemisia/m/sl}%
{*/*/sc}{LGR/artemisia/m/sc}%
{*/*/sco}{LGR/artemisia/m/sco}%
}
\makeatletter
\newcommand*{\rom}[1]{\expandafter\@slowromancap\romannumeral #1@}
\makeatother
\DeclarePairedDelimiterX{\infdivx}[2]{(}{)}{%
#1\;\delimsize\|\;#2%
}
\newcommand{\infdiv}{D\infdivx}
\DeclarePairedDelimiter{\norm}{\left\lVert}{\right\rVert}
\DeclarePairedDelimiter{\ceil}{\left\lceil}{\right\rceil}
\DeclarePairedDelimiter{\floor}{\left\lfloor}{\right\rfloor}
\def\Z{\mathbb Z}
\def\R{\mathbb R}
\def\C{\mathbb C}
\def\N{\mathbb N}
\def\Q{\mathbb Q}
\def\noi{\noindent}
\onehalfspace
\usemintedstyle{bw}
\author{Sandy Urazayev\thanks{University of Kansas (ctu@ku.edu)}}
\date{107; 12021 H.E.}
\title{Homework 7 Oracle\\\medskip
\large MATH 220 Spring 2021}
\hypersetup{
 pdfauthor={Sandy Urazayev},
 pdftitle={Homework 7 Oracle},
 pdfkeywords={},
 pdfsubject={},
 pdfcreator={Emacs 28.0.50 (Org mode 9.4.5)}, 
 pdflang={English}}
\begin{document}

\maketitle
\href{./index.pdf}{[View the PDF version]​}
\section*{Section 3.7}
\label{sec:orgd213da5}
\subsection*{Problem 1 [FOR GRADE]}
\label{sec:org7780792}
We wish to write the two sinusoidal terms as one.
\begin{align*}
	3 \cos 2 t+4 \sin 2 t & =R \cos \left(\omega_{0} t-\delta\right)                                   \\
	                      & =R\left[\cos \omega_{0} t \cos \delta+\sin \omega_{0} t \sin \delta\right] \\
	                      & =(R \cos \delta) \cos \omega_{0} t+(R \sin \delta) \sin \omega_{0} t
\end{align*}

Matching the coefficients, we obtain the following system of equations for
\(\omega_{0}, R\), and \(\delta\).

\begin{align*}
R \cos \delta=3 \quad \quad \quad (1)\\
\omega_{0}=2 \quad \quad \quad (2)\\
R \sin \delta=4 \quad \quad \quad (3)
\end{align*}

Square both sides of the first and third equations

\begin{align*}
R^{2} \cos ^{2} \delta=9 \\
R^{2} \sin ^{2} \delta=16
\end{align*}

and add their respective sides.

\begin{align*}
R^{2} \cos ^{2} \delta+R^{2} \sin ^{2} \delta=9+16 \\
R^{2}\left(\cos ^{2} \delta+\sin ^{2} \delta\right)=25 \\
R^{2}=25 \\
R=5
\end{align*}

Divide the respective sides of equations (1) and (3).

\begin{equation*}
\frac{R \sin \delta}{R \cos \delta}=\frac{4}{3} \quad \rightarrow \quad \tan \delta=\frac{4}{3} \quad \rightarrow \quad \delta=\tan ^{-1} \frac{4}{3}
\end{equation*}

Therefore,

\begin{equation*}
3 \cos 2 t+4 \sin 2 t=5 \cos \left(2 t-\tan ^{-1} \frac{4}{3}\right)
\end{equation*}

\subsection*{Problem 5}
\label{sec:org9b074af}
\begin{align*}
  20 u'' + 400 u' + 3920u &= 0\\
  20r^2 + 400r + 3920 &= 0
\end{align*}
then
\begin{align*}
  r = -10 \pm 4 \sqrt{6} i
\end{align*}
We see that
\begin{align*}
        u(t) =  & C_1 e^{-10t}\cos(4\sqrt{6}t) + C_2 e^{-10t}\sin(4\sqrt{6}t)                   \\
        u'(t) = & 4\sqrt{6}C_1e^{-10t}\sin(4\sqrt{6}t) - 10 C_1 e^{-10t} \cos(4\sqrt{6}t)       \\
                & + 4\sqrt{6} C_2 e^{-10t} \cos(4\sqrt{6}t) - 10 C_2 e^{-10t} \sin (4\sqrt{6}t)
\end{align*}
By knowing that \(u(0)=2\) and \(u'(0)=2\), we find that \(C_1=2\) and
\(C_2=\frac{5}{\sqrt{6}}\). Finally,
\begin{align*}
  u(t) = 2 e^{-10t} \cos(4\sqrt{6}t) + \frac{5}{\sqrt{6}} e^{-10t} \sin(4\sqrt{6}t)
\end{align*}
Therefore
\begin{align*}
  \text{Quasi-frequency} &: 4\sqrt{6}\\
  \text{Quasi-period}    &: \frac{\pi}{2\sqrt{6}}
\end{align*}
\section*{Section 7.1}
\label{sec:org8e96b4a}
\subsection*{Problem 1 [FOR GRADE]}
\label{sec:orgefce265}
   Let \(u=x_{1}\).
$$
x_{1}^{\prime \prime}+0.5 x_{1}^{\prime}+2 x_{1}=0
$$
Finally, let \(x_{2}=x_{1}^{\prime}\).
$$
x_{2}^{\prime}+0.5 x_{2}+2 x_{1}=0
$$
By making these substitutions, the original second-order ODE has become a system of first-order ODEs.
$$
\left\{\begin{array}{l}
x_{1}^{\prime}=x_{2} \\
x_{2}^{\prime}=-2 x_{1}-0.5 x_{2}
\end{array}\right.
$$
\subsection*{Problem 5}
\label{sec:org30bc8b8}
   Let \(u=x_{1}\).
$$
x_{1}^{\prime \prime}+p(t) x_{1}^{\prime}+q(t) x_{1}=g(t), \quad x_{1}(0)=u_{0}, \quad x_{1}^{\prime}(0)=u_{0}^{\prime}
$$
Finally, let \(x_{2}=x_{1}^{\prime}\).
$$
x_{2}^{\prime}+p(t) x_{2}+q(t) x_{1}=g(t), \quad x_{1}(0)=u_{0}, \quad x_{2}(0)=u_{0}^{\prime}
$$
By making these substitutions, the original initial value problem has become a system of first-order ODEs,
$$
\left\{\begin{array}{l}
x_{1}^{\prime}=x_{2} \\
x_{2}^{\prime}=-q(t) x_{1}-p(t) x_{2}+g(t)
\end{array}\right.
$$
subject to the initial conditions,
$$
x_{1}(0)=u_{0} \quad \text { and } \quad x_{2}(0)=u_{0}^{\prime} .
$$
\section*{Section 7.2}
\label{sec:org5864cdb}
\subsection*{Problem 4}
\label{sec:orga6adf18}
   \(\mathbf{A}^{T}\) is the transpose of \(\mathbf{A}, \overline{\mathbf{A}}\) is
the complex conjugate of \(\mathbf{A}\), and
\(\mathbf{A}^{*}=\overline{\mathbf{A}}^{T}\) is the adjoint of \(\mathbf{A}\). 

\begin{equation*}
\mathbf{A}^{T}=\left(\begin{array}{cc}
3-2 i & 2-i \\
1+i & -2+3 i
\end{array}\right)
\end{equation*}

\begin{equation*}
\overline{\mathbf{A}}=\left(\begin{array}{cc}
3+2 i & 1-i \\
2+i & -2-3 i
\end{array}\right)
\end{equation*}

\begin{equation*}
\mathbf{A}^{*}=\left(\begin{array}{cc}
3+2 i & 2+i \\
1-i & -2-3 i
\end{array}\right)
\end{equation*}
\subsection*{Problem 8 [FOR GRADE]}
\label{sec:org8896863}
   Start by calculating the determinant.
$$
\operatorname{det}\left(\begin{array}{rr}
1 & 4 \\
-2 & 3
\end{array}\right)=(1)(3)-(4)(-2)=11
$$
Since it's not zero, an inverse for the given matrix exists.
$$
\left(\begin{array}{rr|rr}
1 & 4 & 1 & 0 \\
-2 & 3 & 0 & 1
\end{array}\right)
$$
The aim is to make the left side of the augmented matrix 1's and 0 's as the right side is now. Since the top left entry is 1 already, we move on to the bottom left entry. To make it zero, multiply both sides of the first row by 2 and add it to the second row.
$$
\left(\begin{array}{cc|cc}
1 & 4 & 1 & 0 \\
0 & 11 & 2 & 1
\end{array}\right)
$$
To make the bottom right entry 1 , divide the bottom row by 11 .
$$
\left(\begin{array}{ll|ll}
1 & 4 & 1 & 0 \\
0 & 1 & \frac{2}{11} & \frac{1}{11}
\end{array}\right)
$$
To make the top right entry 0 , multiply the bottom row by \(-4\) and add it to the first row.
$$
\left(\begin{array}{cc|cc}
1 & 0 & \frac{3}{11} & -\frac{4}{11} \\
0 & 1 & \frac{2}{11} & \frac{1}{11}
\end{array}\right)
$$
Therefore, the inverse matrix is
$$
\left(\begin{array}{cc}
\frac{3}{11} & -\frac{4}{11} \\
\frac{2}{11} & \frac{1}{11}
\end{array}\right) .
$$
\end{document}