% Created 2021-04-19 Mon 20:13
% Intended LaTeX compiler: xelatex
\documentclass[12pt]{article}
\usepackage{graphicx}
\usepackage{grffile}
\usepackage{longtable}
\usepackage{wrapfig}
\usepackage{rotating}
\usepackage[normalem]{ulem}
\usepackage{amsmath}
\usepackage{textcomp}
\usepackage{amssymb}
\usepackage{capt-of}
\usepackage{hyperref}
\usepackage{minted}
\usepackage{amsmath}
\usepackage{amssymb}
\usepackage{setspace}
\usepackage{subcaption}
\usepackage{mathtools}
\usepackage{xfrac}
\usepackage[margin=1in]{geometry}
\usepackage[utf8]{inputenc}
\usepackage{color}
\usepackage{epsf}
\usepackage{tikz}
\usepackage{graphicx}
\usepackage{pslatex}
\usepackage{hyperref}

\usepackage{concmath}
\usepackage[OT1]{fontenc}

\usepackage{textgreek}
\renewcommand*{\textgreekfontmap}{%
{phv/*/*}{LGR/neohellenic/*/*}%
{*/b/n}{LGR/artemisia/b/n}%
{*/bx/n}{LGR/artemisia/bx/n}%
{*/*/n}{LGR/artemisia/m/n}%
{*/b/it}{LGR/artemisia/b/it}%
{*/bx/it}{LGR/artemisia/bx/it}%
{*/*/it}{LGR/artemisia/m/it}%
{*/b/sl}{LGR/artemisia/b/sl}%
{*/bx/sl}{LGR/artemisia/bx/sl}%
{*/*/sl}{LGR/artemisia/m/sl}%
{*/*/sc}{LGR/artemisia/m/sc}%
{*/*/sco}{LGR/artemisia/m/sco}%
}
\makeatletter
\newcommand*{\rom}[1]{\expandafter\@slowromancap\romannumeral #1@}
\makeatother
\DeclarePairedDelimiterX{\infdivx}[2]{(}{)}{%
#1\;\delimsize\|\;#2%
}
\newcommand{\infdiv}{D\infdivx}
\DeclarePairedDelimiter{\norm}{\left\lVert}{\right\rVert}
\DeclarePairedDelimiter{\ceil}{\left\lceil}{\right\rceil}
\DeclarePairedDelimiter{\floor}{\left\lfloor}{\right\rfloor}
\def\Z{\mathbb Z}
\def\R{\mathbb R}
\def\C{\mathbb C}
\def\N{\mathbb N}
\def\Q{\mathbb Q}
\def\noi{\noindent}
\onehalfspace
\usemintedstyle{bw}
\author{Sandy Urazayev\thanks{University of Kansas (ctu@ku.edu)}}
\date{93; 12021 H.E.}
\title{Homework 5 Oracle\\\medskip
\large MATH 220 Spring 2021}
\hypersetup{
 pdfauthor={Sandy Urazayev},
 pdftitle={Homework 5 Oracle},
 pdfkeywords={},
 pdfsubject={},
 pdfcreator={Emacs 28.0.50 (Org mode 9.4.5)}, 
 pdflang={English}}
\begin{document}

\maketitle
\href{./index.pdf}{[View the PDF version]​}

\section*{Section 3.1}
\label{sec:orga730354}
\subsection*{Problem 9}
\label{sec:org8a2702f}
\begin{equation*}
  y'' + 3y' = 0, \quad y(0) = -2, \quad y'(0) = 3
\end{equation*}
Since this is a linear homogeneous constant-coefficient ODE, the solution is
of the form \(y = e^{rt}\)
\begin{equation*}
  y=e^{rt} \quad \implies \quad y' = re^{rt} \quad \implies \quad y'' = r^2 e^{rt}
\end{equation*}
Substitute those expressions into the ODE
\begin{equation*}
  r^2 e^{rt} + 3(re^{rt}) = 0
\end{equation*}
Divide both sides by \(e^{rt}\)
\begin{equation*}
  r^2 + 3r = 0
\end{equation*}
Roots of this polynomial are \(r_0 = -3\) and \(r_1 = 0\). Two solutions to the
ODE are \(y=e^{-3t}\) and \(y=e^0=1\). Therefore, the general solution is
\begin{equation*}
  y(t) = C_1 e^{-3t} + C_2
\end{equation*}
Differentiating \(y\) gives us
\begin{equation*}
  y'(t) = -3C_1 e^{-3t}
\end{equation*}
Now, we can determine our constants by applying the two initial conditions we
know
\begin{equation*}
  \begin{cases}
    y(0) = C_1 + C_2 = -2\\
    y'(0) = -3C_1 = 3
  \end{cases}
\end{equation*}
Therefore \(C_1 = -1\) and \(C_2 = -1\), therefore
\begin{equation*}
  y(t) = -e^{-3t} - 1
\end{equation*}
This solution converges to \(-1\) as \(t \to \infty\).
\subsection*{Problem 13 [FOR GRADE]}
\label{sec:org901e35e}
Find a differential equation whose general solution is
\begin{equation*}
  y=c_{1} e^{2 t}+c_{2} e^{-3 t}
\end{equation*}
We see the roots are \(r_0 = -3\) and \(r_1 = 2\). Alternatively, you can make a
set of solutions, and call it \(r = \{-3,2\}\). So
\begin{align*}
  (r+3)(r-2)&=0 \\
  \implies r^2 + r - 6 &= 0
\end{align*}
Multiply both sides by \(e^{rt}\)
\begin{align*}
  r^2e^{rt} + re^{rt} - 6e^{rt} = 0
\end{align*}
Therefore, the differential equation is
\begin{align*}
  y'' + y' - 6y = 0
\end{align*}
\subsection*{Problem 16}
\label{sec:orga604cd2}
This is a linear homogeneous constant-coefficient ODE, apply the same method
as in Problem 9. Find that \(r = \{-1, 2\}\) and the general solution is
\begin{align*}
  y(t) = C_1 e^{-t} + C_2 e^{2t}
\end{align*}
The derivative would be
\begin{align*}
  y'(t) = -C_1 e^{-t} + 2 C_2 e^{2t}
\end{align*}
Let us solve the initial conditions
\begin{align*}
  \begin{cases}
    y(0) = C_1 + C_2 = \alpha\\
    y'(0) = -C_1 + 2C_2 = 2
  \end{cases}
  \implies \begin{cases}
    C_1 = \frac{2}{3}(\alpha -1)\\
    C_2 = \frac{1}{3}(\alpha +2)
  \end{cases}
\end{align*}
Therefore,
\begin{align*}
  y(t) = \frac{2}{3}(\alpha -1)e^{-t} + \frac{1}{3}(\alpha +2)e^{2t}
\end{align*}
We can see that if \(t \to \infty\), then \(y \to \infty\). Therefore, set
\(\alpha = -2\).
\subsection*{Problem 19}
\label{sec:org3716464}
\begin{align*}
  y'' + 5y' + 6y = 9, \quad y(0) = 2, \quad y'(0) = \beta,
\end{align*}
where \(\beta > 0\).
\subsubsection*{Part (a)}
\label{sec:org9b24bfc}
This is a linear homogeneous constant-coefficient ODE, find that
\(r = {-\frac{1}{2}, \frac{1}{2}}\). The two solutions are
\begin{align*}
  y(t) = C_1 e^{-\frac{t}{2}} + C_2 e^{\frac{t}{2}}
\end{align*}
Then
\begin{align*}
  y'(t) = -\frac{C_1}{2} e^{-\frac{t}{2}} + \frac{C_2}{2} e^{\frac{t}{2}}
\end{align*}
Solve
\begin{align*}
  \begin{cases}
    y(0) = C_1 + C_2 = 2\\
    y'(0) = -\frac{C_1}{2} + \frac{C_2}{2} = \beta
  \end{cases} \implies
  \begin{cases}
    C_1 = 1 - \beta\\
    C_2 = 1 + \beta
  \end{cases}
\end{align*}
Finally,
\begin{align*}
  y(t) = (1 - \beta)e^{-\frac{t}{2}} + (1+\beta)e^{\frac{t}{2}}
\end{align*}
To prevent the solution from going to the infinity and beyond, set
\(\beta=-1\).
\subsubsection*{Part (b, c, d)}
\label{sec:org9427665}
See Professor Van Vleck's notes on this problem.
\subsection*{Problem 21 [FOR GRADE]}
\label{sec:org9d17c3d}
\begin{align*}
  ay'' + by' + cy = 0,
\end{align*}
where \(a, b, c \in \mathbb{R}\) and \(a > 0\).

This is yet again another linear homogeneous constant-coefficient ODE. Find
that
  \begin{align*}
a\left(r^{2} e^{r t}\right)+b\left(r e^{r t}\right)+c\left(e^{r t}\right)=0
\end{align*}
Divide both sides by \(e^{r t}\)
\begin{align*}
a r^{2}+b r+c=0 \\
\implies r=\frac{-b \pm \sqrt{b^{2}-4 a c}}{2 a}
\end{align*}
\subsubsection*{Part (a)}
\label{sec:org17de46e}
For the roots to be real, different and negative, \(b>0\) and \(0 < c < \frac{b^2}{4a}\).
\subsubsection*{Part (b)}
\label{sec:org8c1fb50}
For the roots to be real with opposite signs, \(c < 0\).
\subsubsection*{Part (c)}
\label{sec:orge4ed651}
For the roots to be real, different and positive, \(b<0\) and
\(0 < c < \frac{b^2}{4a}\).
\section*{Section 3.2}
\label{sec:org25d57ae}
\subsection*{Problem 5}
\label{sec:orgae0b392}
The Wronskian of these two functions is
\begin{align*}
W &=\left|\begin{array}{cc}
\cos ^{2} \theta & 1+\cos 2 \theta \\
\frac{d}{d \theta}\left(\cos ^{2} \theta\right) & \frac{d}{d \theta}(1+\cos 2 \theta)
\end{array}\right| \\
&=\left|\begin{array}{cc}
\cos ^{2} \theta & 1+\cos 2 \theta \\
2 \cos \theta(-\sin \theta) & -2 \sin 2 \theta
\end{array}\right| \\
&=\cos ^{2} \theta(-2 \sin 2 \theta)-(1+\cos 2 \theta)[2 \cos \theta(-\sin \theta)] \\
&=-2 \cos ^{2} \theta \sin 2 \theta+2 \sin \theta \cos \theta(1+\cos 2 \theta) \\
&=-2 \cos ^{2} \theta(2 \sin \theta \cos \theta)+2 \sin \theta \cos \theta\left(1+2 \cos ^{2} \theta-1\right) \\
&=-4 \cos ^{2} \theta \sin \theta \cos \theta+4 \sin \theta \cos \theta \cos ^{2} \theta \\
&=0
\end{align*}

\subsection*{Problem 22 [FOR GRADE]}
\label{sec:orgbbb5ea7}
\begin{align*}
  y'' - y' - 2y = 0
\end{align*}

\textbf{Note:} Solutions for this problem are based on Jock's solutions.

\subsubsection*{Part (a)}
\label{sec:org80dd270}
Calculate \(W\left(y_{1}, y_{2}\right)\) the Wronskian of \(y_{1}\) and
\(y_{2}\).

\begin{align*}
W\left(y_{1}, y_{2}\right) &=\left|\begin{array}{ll}
y_{1} & y_{2} \\
y_{1}^{\prime} & y_{2}^{\prime}
\end{array}\right| \\
&=\left|\begin{array}{cc}
e^{-t} & e^{2 t} \\
-e^{-t} & 2 e^{2 t}
\end{array}\right| \\
&=e^{-t}\left(2 e^{2 t}\right)-e^{2 t}\left(-e^{-t}\right) \\
&=2 e^{t}+e^{t} \\
&=3 e^{t}
\end{align*}

Since \(W\left(y_{1}, y_{2}\right) \neq 0, y_{1}\) and \(y_{2}\) form a fundamental
set of solutions.

\subsubsection*{Part (b)}
\label{sec:org36c72d7}

Check that \(y_{3}\) is a solution of the ODE.

\begin{array}{c}
y_{3}^{\prime \prime}-y_{3}^{\prime}-2 y_{3} \stackrel{?}{=} 0 \\
\frac{d^{2}}{d t^{2}}\left(-2 e^{2 t}\right)-\frac{d}{d t}\left(-2 e^{2 t}\right)-2\left(-2 e^{2 t}\right) \stackrel{?}{=} 0 \\
\left(-8 e^{2 t}\right)-\left(-4 e^{2 t}\right)-2\left(-2 e^{2 t}\right) \stackrel{?}{=} 0 \\
-8 e^{2 t}+4 e^{2 t}+4 e^{2 t} \stackrel{?}{=} 0 \\
0=0
\end{array}

Now check that \(y_{4}=e^{-t}+2 e^{2 t}\) is a solution of the ODE.

\begin{array}{c}
y_{4}^{\prime \prime}-y_{4}^{\prime}-2 y_{4} \stackrel{?}{=} 0 \\
\frac{d^{2}}{d t^{2}}\left(e^{-t}+2 e^{2 t}\right)-\frac{d}{d t}\left(e^{-t}+2 e^{2 t}\right)-2\left(e^{-t}+2 e^{2 t}\right) \stackrel{?}{=} 0 \\
\left(e^{-t}+8 e^{2 t}\right)-\left(-e^{-t}+4 e^{2 t}\right)-2\left(e^{-t}+2 e^{2 t}\right) \stackrel{?}{=} 0 \\
e^{-^{\ell}}+8 e^{2 t}+e^{-}-4 e^{2 t}-2 e^{-}-4 e^{2 t} \stackrel{?}{=} 0 \\
0=0
\end{array}

Now check that \(y_{5}=2 y_{1}(t)-2 y_{3}(t)=2 e^{-t}-2\left(-2 e^{2 t}\right)=2
e^{-t}+4 e^{2 t}\) is a solution of the ODE. 

\begin{array}{c}
y_{5}^{\prime \prime}-y_{5}^{\prime}-2 y_{5} \stackrel{?}{=} 0 \\
\frac{d^{2}}{d t^{2}}\left(2 e^{-t}+4 e^{2 t}\right)-\frac{d}{d t}\left(2 e^{-t}+4 e^{2 t}\right)-2\left(2 e^{-t}+4 e^{2 t}\right) \stackrel{?}{=} 0 \\
\left(2 e^{-t}+16 e^{2 t}\right)-\left(-2 e^{-t}+8 e^{2 t}\right)-2\left(2 e^{-t}+4 e^{2 t}\right) \stackrel{?}{=} 0 \\
2 e^{-}+16 e^{2 t}+2 e^{-}-8 e^{2 t}-4 e^{-}-8 e^{2 t} \stackrel{?}{=} 0 \\
0=0
\end{array}

\subsubsection*{Part (c)}
\label{sec:org1d352ea}

Calculate \(W\left(y_{1}, y_{3}\right)\), the Wronskian of \(y_{1}\) and \(y_{3}\).

\begin{aligned}
W\left(y_{1}, y_{3}\right) &=\left|\begin{array}{ll}
y_{1} & y_{3} \\
y_{1}^{\prime} & y_{3}^{\prime}
\end{array}\right| \\
&=\left|\begin{array}{cc}
e^{-t} & -2 e^{2 t} \\
-e^{-t} & -4 e^{2 t}
\end{array}\right| \\
&=e^{-t}\left(-4 e^{2 t}\right)-\left(-2 e^{2 t}\right)\left(-e^{-t}\right) \\
&=-4 e^{t}-2 e^{t} \\
&=-6 e^{t}
\end{aligned}

Since \(W\left(y_{1}, y_{3}\right) \neq 0, y_{1}\) and \(y_{3}\) form a fundamental set of solutions.

Now calculate \(W\left(y_{2}, y_{3}\right)\), the Wronskian of \(y_{2}\) and \(y_{3}\)

\begin{aligned}
W\left(y_{2}, y_{3}\right) &=\left|\begin{array}{ll}
y_{2} & y_{3} \\
y_{2}^{\prime} & y_{3}^{\prime}
\end{array}\right| \\
&=\left|\begin{array}{cc}
e^{2 t} & -2 e^{2 t} \\
2 e^{2 t} & -4 e^{2 t}
\end{array}\right| \\
&=e^{2 t}\left(-4 e^{2 t}\right)-\left(-2 e^{2 t}\right)\left(2 e^{2 t}\right) \\
&=-4 e^{4 t}+4 e^{4 t} \\
&=0
\end{aligned}

Since \(W\left(y_{2}, y_{3}\right)=0, y_{2}\) and \(y_{3}\) do not form a
fundamental set of solutions. Now calculate \(W\left(y_{1}, y_{4}\right)\), the
Wronskian of \(y_{1}\) and \(y_{4}\) 

\begin{aligned}
W\left(y_{1}, y_{4}\right) &=\left|\begin{array}{ll}
y_{1} & y_{4} \\
y_{1}^{\prime} & y_{4}^{\prime}
\end{array}\right| \\
&=\left|\begin{array}{cc}
e^{-t} & e^{-t}+2 e^{2 t} \\
-e^{-t} & -e^{-t}+4 e^{2 t}
\end{array}\right| \\
&=e^{-t}\left(-e^{-t}+4 e^{2 t}\right)-\left(e^{-t}+2 e^{2 t}\right)\left(-e^{-t}\right) \\
&=-e^{-2 t}+4 e^{t}+e^{-2 t}+2 e^{t} \\
&=6 e^{t}
\end{aligned}

Since \(W\left(y_{1}, y_{4}\right) \neq 0, y_{1}\) and \(y_{4}\) form a fundamental
set of solutions. Now calculate \(W\left(y_{4}, y_{5}\right)\), the Wronskian of
\(y_{4}\) and \(y_{5}\). 

\begin{aligned}
W\left(y_{4}, y_{5}\right) &=\left|\begin{array}{ll}
y_{4} & y_{5} \\
y_{4}^{\prime} & y_{5}^{\prime}
\end{array}\right| \\
&=\left|\begin{array}{cc}
e^{-t}+2 e^{2 t} & 2 e^{-t}+4 e^{2 t} \\
-e^{-t}+4 e^{2 t} & -2 e^{-t}+8 e^{2 t}
\end{array}\right| \\
&=\left(e^{-t}+2 e^{2 t}\right)\left(-2 e^{-t}+8 e^{2 t}\right)-\left(2 e^{-t}+4 e^{2 t}\right)\left(-e^{-t}+4 e^{2 t}\right) \\
&=-2 e^{-2 t}+8 e^{t}-4 e^{t}+16 e^{4 t}-\left(-2 e^{-2 t}+8 e^{t}-4 e^{t}+16 e^{4 t}\right) \\
&=0
\end{aligned}

Since \(W\left(y_{4}, y_{5}\right)=0, y_{4}\) and \(y_{5}\) do not form a fundamental set of solutions.
\subsection*{Problem 24}
\label{sec:org6ad14aa}
\begin{align*}
  (\cos t)y'' +(\sin t)y' -ty = 0
\end{align*}
Then
\begin{align*}
  y'' + \frac{\sin t}{\cos t} - \frac{t}{\cos t}y = 0
\end{align*}
so
\begin{align*}
  p(t) = \tan t
\end{align*}
Then
\begin{align*}
  W = C \exp\left(-\int \tan t dt \right)
\end{align*}
By Abel's Theorem
\begin{align*}
  W = C \exp \left( \ln (cos t) \right) \implies W = C\times \cos t
\end{align*}
\subsection*{Problem 31}
\label{sec:orgfdd0246}
The equation
\begin{align*}
        P(x) y^{\prime \prime}+Q(x) y^{\prime}+R(x) y=0
\end{align*}
is said to be exact if it can be written in the form
\begin{align*}
	\left(P(x) y^{\prime}\right)^{\prime}+(f(x) y)^{\prime}=0
\end{align*}
where \(f(x)\) is to be determined in terms of \$P(x), Q(x),\$ and \(R(x)\) The latter equation can be integrated once immediately, resulting in a first-order linear equation for \(y\) that can be solved as in Section 2.1. By equating the coefficients of the preceding equations and then eliminating \(f(x)\), show that a necessary condition for exactness is
\begin{align*}
	P^{\prime \prime}(x)-Q^{\prime}(x)+R(x)=0
\end{align*}
It can be shown that this is also a sufficient condition.
\end{document}