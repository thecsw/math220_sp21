% Created 2021-04-21 Wed 20:59
% Intended LaTeX compiler: xelatex
\documentclass[12pt]{article}
\usepackage{graphicx}
\usepackage{grffile}
\usepackage{longtable}
\usepackage{wrapfig}
\usepackage{rotating}
\usepackage[normalem]{ulem}
\usepackage{amsmath}
\usepackage{textcomp}
\usepackage{amssymb}
\usepackage{capt-of}
\usepackage{hyperref}
\usepackage{minted}
\usepackage{amsmath}
\usepackage{amssymb}
\usepackage{setspace}
\usepackage{subcaption}
\usepackage{mathtools}
\usepackage{xfrac}
\usepackage[margin=1in]{geometry}
\usepackage{marginnote}
\usepackage[utf8]{inputenc}
\usepackage{color}
\usepackage{epsf}
\usepackage{tikz}
\usepackage{graphicx}
\usepackage{pslatex}
\usepackage{hyperref}

\usepackage{beton}
\usepackage{euler}
\usepackage[OT1]{fontenc}

\usepackage{textgreek}
\renewcommand*{\textgreekfontmap}{%
{phv/*/*}{LGR/neohellenic/*/*}%
{*/b/n}{LGR/artemisia/b/n}%
{*/bx/n}{LGR/artemisia/bx/n}%
{*/*/n}{LGR/artemisia/m/n}%
{*/b/it}{LGR/artemisia/b/it}%
{*/bx/it}{LGR/artemisia/bx/it}%
{*/*/it}{LGR/artemisia/m/it}%
{*/b/sl}{LGR/artemisia/b/sl}%
{*/bx/sl}{LGR/artemisia/bx/sl}%
{*/*/sl}{LGR/artemisia/m/sl}%
{*/*/sc}{LGR/artemisia/m/sc}%
{*/*/sco}{LGR/artemisia/m/sco}%
}
\makeatletter
\newcommand*{\rom}[1]{\expandafter\@slowromancap\romannumeral #1@}
\makeatother
\DeclarePairedDelimiterX{\infdivx}[2]{(}{)}{%
#1\;\delimsize\|\;#2%
}
\newcommand{\infdiv}{D\infdivx}
\DeclarePairedDelimiter{\norm}{\left\lVert}{\right\rVert}
\DeclarePairedDelimiter{\ceil}{\left\lceil}{\right\rceil}
\DeclarePairedDelimiter{\floor}{\left\lfloor}{\right\rfloor}
\def\Z{\mathbb Z}
\def\R{\mathbb R}
\def\C{\mathbb C}
\def\N{\mathbb N}
\def\Q{\mathbb Q}
\def\noi{\noindent}
\onehalfspace
\usemintedstyle{bw}
\author{Sandy Urazayev\thanks{University of Kansas (ctu@ku.edu)}}
\date{93; 12021 H.E.}
\title{Homework 6 Oracle\\\medskip
\large MATH 220 Spring 2021}
\hypersetup{
 pdfauthor={Sandy Urazayev},
 pdftitle={Homework 6 Oracle},
 pdfkeywords={},
 pdfsubject={},
 pdfcreator={Emacs 28.0.50 (Org mode 9.4.5)}, 
 pdflang={English}}
\begin{document}

\maketitle
\href{./index.pdf}{[View the PDF version]​}

\section*{Section 3.3}
\label{sec:orgb309973}
\subsection*{Problem 7}
\label{sec:org634be96}
\begin{align*}
  y'' + 2y' + 2y = 0
\end{align*}
Find that \(r = \{-1 - i, -1 + i\}\). Then
\begin{align*}
  y(t) = C_1 e^{-t} \cos (t) + C_2 e^{-t} \sin (t)
\end{align*}
\subsection*{Problem 13 [FOR GRADE]}
\label{sec:org918d984}
\begin{align*}
  y'' - 2y' + 5y = 0, \quad y(\pi/2) = 0, \quad y'(\pi / 2) = 2
\end{align*}
Find that \(r = \{1+2i,1-2i\}\). Then
\begin{align*}
  y(t) = C_1 e^t \cos (2t) + C_2 e^t \sin (2t)
\end{align*}
Let us apply the initial conditions
\begin{align*}
  \begin{cases}
    y(\pi/2)=0\\
    y'(\pi/2)=2
    \end{cases} \implies
  \begin{cases}
    C_1 = 0\\
    C_2 = -\frac{1}{e^{\pi/2}}
    \end{cases}
\end{align*}
Finally,
\begin{align*}
  y(t) = -\frac{1}{e^{\pi/2}} e^t \sin (2t)
\end{align*}
\subsection*{Problem 17}
\label{sec:org8503826}
\begin{align*}
  5u'' + 2u' + 7u = 0, \quad u(0) = 2, \quad u'(0) = 1
\end{align*}
\subsubsection*{Part (a)}
\label{sec:org11ba0f1}
Find that
\(r = \{-\frac{1}{5} - i\frac{\sqrt{34}}{5}, -\frac{1}{5} +
    i\frac{\sqrt{34}}{5}\}\). Then
\begin{align*}
  \begin{cases}
    u(0) = 2\\
    u'(0) = 1
  \end{cases} \implies
  \begin{cases}
    C_1 = 2\\
    C_2 = \frac{7}{\sqrt{34}}
  \end{cases}
\end{align*}
Finally,
\begin{align*}
  u(t) = 2 e^{-t/5} \cos\left(\frac{\sqrt{34}t}{5}\right) + \frac{7}{\sqrt{34}} e^{-t/5} \sin\left(\frac{\sqrt{34}t}{5}\right)
\end{align*}
\subsubsection*{Part (b)}
\label{sec:org3b9d47e}
\begin{align*}
  T \approx 14.512
\end{align*}
\subsection*{Problem 19}
\label{sec:org49c54e7}
\begin{aligned} W\left(e^{\lambda t} \cos \mu t, e^{\lambda t} \sin \mu
        t\right) & =\left|\begin{array}{cc}e^{\lambda t} \cos \mu t & e^{\lambda t}
                \sin \mu t                                   \\ \frac{d}{d t}\left(e^{\lambda t} \cos \mu t\right) & \frac{d}{d
                        t}\left(e^{\lambda t} \sin \mu t\right)\end{array}\right|
        \\ &=\left|\begin{array}{cc}e^{\lambda t} \cos \mu t & e^{\lambda t} \sin \mu
                t                                                     \\ \lambda e^{\lambda t} \cos \mu t-\mu e^{\lambda t} \sin \mu t \quad
                \lambda e^{\lambda t} \sin \mu t+\mu e^{\lambda t} \cos \mu
                t\end{array}\right| \\ &=e^{\lambda t} \cos \mu t\left(\lambda e^{\lambda t}
        \sin \mu t+\mu e^{\lambda t} \cos \mu t\right)-e^{\lambda t} \sin \mu
        t\left(\lambda e^{\lambda t} \cos \mu t-\mu e^{\lambda t} \sin \mu t\right)
        \\ &=\lambda e^{2 \lambda t} \cos \mu t \sin \mu t+\mu e^{2
                        \lambda t} \cos ^{2} \mu t\lambda e^{2 \lambda t} \cos \mu t
        \sin \mu t+\mu e^{2 \lambda t} \sin ^{2} \mu t     \\ &=\mu e^{2 \lambda
        t}\left(\cos ^{2} \mu t+\sin ^{2} \mu t\right)     \\ &=\mu e^{2 \lambda t}
\end{aligned}

\section*{Section 3.4}
\label{sec:orgb54aea4}
\subsection*{Problem 1}
\label{sec:orga6c2ee2}
\begin{align*}
  y'' - 2y' + y = 0
\end{align*}
Then
\begin{align*}
  r^2 - 2r + 1 = 0
\end{align*}
So \(r = 1\), finally
\begin{align*}
  y(t) = C_1 e^t + C_2 t e^t
\end{align*}
\subsection*{Problem 8 [FOR GRADE]}
\label{sec:org3c43c16}
\begin{align*}
  2y'' + 2y' + y = 0
\end{align*}
Then
\begin{align*}
  2r^2 + 2r + 1 = 0
\end{align*}
So \(r = \{-1/2 - 1/2i, -1/2 + 1/2i\}\). Finally,
\begin{align*}
  y(t) = C_1 e^{-t/2} \cos \left(\frac{1}{2}t\right) + C_2 e^{-t/2} \sin \left(\frac{1}{2}t\right)
\end{align*}
\subsection*{Problem 11}
\label{sec:org2b91e5c}
\begin{align*}
  y'' + 4y' + 4y = 0, \quad y(-1)=2, \quad y'(-1) = 1
\end{align*}
Then \(r=\{-2\}\). So
\begin{align*}
  y(t) = C_1 e^{-2t} + C_2 t e^{-2t}
\end{align*}
Taking the derivative
\begin{align*}
  y'(t) = C_1 e^{-2t} - 2 C_1t e^{-2t} - 2 C_2 e^{-2t}
\end{align*}
Applying the initial conditions
\begin{align*}
  \begin{cases}
    y(-1) = -C_1e^2 + C_2 e^2 = 2\\
    y'(-1) = C_1 e^2 + 2 C_1 e^2 - 2 C_2 e^2 = 1
  \end{cases} \implies
  \begin{cases}
    C_1 = \frac{5}{e^2}\\
    C_2 = \frac{7}{e^2}
  \end{cases}
\end{align*}
Therefore
\begin{align*}
  y(t) = \frac{5}{e^2}te^{-2t} + \frac{7}{e^2}e^{-2t}
\end{align*}
As \(t \to \infty\), \(y \to 0\).
\subsection*{Problem 18 [FOR GRADE]}
\label{sec:orgf9a36c5}
\begin{align*}
  t^2 y'' - 4 t y' + 6 y = 0, \quad t > 0; \quad y_1(t) = t^2
\end{align*}
Let us apply the method of reduction
Let the general solution be of the form
\begin{align*}
  y(t) = v(t) y_1(t) = t^2v(t)
\end{align*}
Let us find the derivatives
\begin{align*}
  y_2'(t) &= t^2v'(t) + 2tv(t)\\
  y_2''(t) &= t^2v''(t) + 2tv'(t) + 2tv'(t) + 2v(t)
\end{align*}
Let us substitute the above into our original ODE
\begin{align*}
  t^2 (t^2v''(t) + 2tv'(t) + 2tv'(t) + 2v(t)) -4t (t^2v'(t) + 2tv(t)) + 6(t^2v(t)) = 0
\end{align*}
Simplifying yields
\begin{align*}
  t^4 v''(t) + 3t^3v'(t) + t^2v(t) - 4t^3v'(t) -8t^2v(t) + 6t^2v(t) = 0\\
  t^4v''(t) = 0
\end{align*}
Then
\begin{align*}
  v''(t) = 0
\end{align*}
Integrate both sides
\begin{align*}
  v'(t) = C_1\\
  v(t)  = C_1 t + C_2
\end{align*}
Finally, the general solution is
\begin{align*}
  y(t) = (C_1t + C_2)t^2 = C_1t^3 + C_2t^2 = C_1 y_1(t) + C_2 y_2(t)
\end{align*}
and the second solution is \(y_2(t) = t^3\).
\end{document}