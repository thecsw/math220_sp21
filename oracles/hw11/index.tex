% Created 2021-05-08 Sat 22:02
% Intended LaTeX compiler: xelatex
\documentclass[12pt]{article}
\usepackage{graphicx}
\usepackage{grffile}
\usepackage{longtable}
\usepackage{wrapfig}
\usepackage{rotating}
\usepackage[normalem]{ulem}
\usepackage{amsmath}
\usepackage{textcomp}
\usepackage{amssymb}
\usepackage{capt-of}
\usepackage{hyperref}
\usepackage{minted}
\usepackage{amsmath}
\usepackage{amssymb}
\usepackage{setspace}
\usepackage{subcaption}
\usepackage{mathtools}
\usepackage{xfrac}
\usepackage[margin=1in]{geometry}
\usepackage{marginnote}
\usepackage[utf8]{inputenc}
\usepackage{color}
\usepackage{epsf}
\usepackage{tikz}
\usepackage{graphicx}
\usepackage{pslatex}
\usepackage{hyperref}

\usepackage{beton}
\usepackage{euler}
\usepackage[OT1]{fontenc}

\usepackage{textgreek}
\renewcommand*{\textgreekfontmap}{%
{phv/*/*}{LGR/neohellenic/*/*}%
{*/b/n}{LGR/artemisia/b/n}%
{*/bx/n}{LGR/artemisia/bx/n}%
{*/*/n}{LGR/artemisia/m/n}%
{*/b/it}{LGR/artemisia/b/it}%
{*/bx/it}{LGR/artemisia/bx/it}%
{*/*/it}{LGR/artemisia/m/it}%
{*/b/sl}{LGR/artemisia/b/sl}%
{*/bx/sl}{LGR/artemisia/bx/sl}%
{*/*/sl}{LGR/artemisia/m/sl}%
{*/*/sc}{LGR/artemisia/m/sc}%
{*/*/sco}{LGR/artemisia/m/sco}%
}
\makeatletter
\newcommand*{\rom}[1]{\expandafter\@slowromancap\romannumeral #1@}
\makeatother
\DeclarePairedDelimiterX{\infdivx}[2]{(}{)}{%
#1\;\delimsize\|\;#2%
}
\newcommand{\infdiv}{D\infdivx}
\DeclarePairedDelimiter{\norm}{\left\lVert}{\right\rVert}
\DeclarePairedDelimiter{\ceil}{\left\lceil}{\right\rceil}
\DeclarePairedDelimiter{\floor}{\left\lfloor}{\right\rfloor}
\def\Z{\mathbb Z}
\def\R{\mathbb R}
\def\C{\mathbb C}
\def\N{\mathbb N}
\def\Q{\mathbb Q}
\def\noi{\noindent}
\onehalfspace
\usemintedstyle{bw}
\author{Sandy Urazayev\thanks{University of Kansas (ctu@ku.edu)}}
\date{128; 12021 H.E.}
\title{Homework 11 Oracle\\\medskip
\large MATH 220 Spring 2021}
\hypersetup{
 pdfauthor={Sandy Urazayev},
 pdftitle={Homework 11 Oracle},
 pdfkeywords={},
 pdfsubject={},
 pdfcreator={Emacs 28.0.50 (Org mode 9.4.5)}, 
 pdflang={English}}
\begin{document}

\maketitle
\href{./index.pdf}{View the PDF version​}

Solution to these problems are provided by STEM Jock.

\section*{Section 6.2}
\label{sec:org7d2a14c}

\subsection*{Problem 3}
\label{sec:org90253ae}

\subsubsection*{Statement}
\label{sec:org6e3a9c0}
   \begin{equation*}
F(s)=\frac{2}{s^{2}+3 s-4}
\end{equation*}

\subsubsection*{Solution}
\label{sec:orgcc875f9}
Factor the denominator.

    \begin{aligned}
F(s) &=\frac{2}{s^{2}+3 s-4} \\
&=\frac{2}{(s+4)(s-1)} \\
&=\frac{2 / 5}{s-1}-\frac{2 / 5}{s+4}
\end{aligned}

Take the inverse Laplace transform now to get \(f(t)\).

$$
f(t)=\frac{2}{5} e^{t}-\frac{2}{5} e^{-4 t}
$$

\subsection*{Problem 11}
\label{sec:org0165227}

\subsubsection*{Statement}
\label{sec:org3f2fa20}
$$
    y^{\prime \prime}-2 y^{\prime}+4 y=0 ; \quad y(0)=2, \quad y^{\prime}(0)=0
 $$

\subsubsection*{Solution}
\label{sec:org4770a33}
    Because the ODE is linear, the Laplace transform can be applied to solve it. The Laplace transform of a function \(y(t)\) is defined here as
$$
Y(s)=\mathcal{L}\{y(t)\}=\int_{0}^{\infty} e^{-s t} y(t) d t
$$
Consequently, the first and second derivatives transform as follows.


\begin{aligned}
\mathcal{L}\left\{\frac{d y}{d t}\right\} &=s Y(s)-y(0) \\
\mathcal{L}\left\{\frac{d^{2} y}{d t^{2}}\right\} &=s^{2} Y(s)-s y(0)-y^{\prime}(0)
\end{aligned}


Apply the Laplace transform to both sides of the ODE.
$$
\mathcal{L}\left\{y^{\prime \prime}-2 y^{\prime}+4 y\right\}=\mathcal{L}\{0\}
$$
Use the fact that the transform is a linear operator.

\begin{array}{c}
\mathcal{L}\left\{y^{\prime \prime}\right\}-2 \mathcal{L}\left\{y^{\prime}\right\}+4 \mathcal{L}\{y\}=0 \\
{\left[s^{2} Y(s)-s y(0)-y^{\prime}(0)\right]-2[s Y(s)-y(0)]+4 Y(s)=0}
\end{array}

Plug in the initial conditions, \(y(0)=2\) and \(y^{\prime}(0)=0\).

$$
\left[s^{2} Y(s)-2 s\right]-2[s Y(s)-2]+4 Y(s)=0
$$

As a result of applying the Laplace transform, the ODE has reduced to an algebraic equation for
  Solution \(Y\) , the transformed solution.

  \begin{equation*}
\begin{array}{c}
s^{2} Y(s)-2 s Y(s)+4 Y(s)-2 s+4=0 \\
\left(s^{2}-2 s+4\right) Y(s)=2 s-4
\end{array}
\end{equation*}

\begin{aligned}
Y(s) &=\frac{2 s-4}{s^{2}-2 s+4} \\
&=\frac{2 s-4}{s^{2}-2 s+1+4-1} \\
&=\frac{2 s-4}{(s-1)^{2}+3} \\
&=\frac{2 s-2-4+2}{(s-1)^{2}+3} \\
&=\frac{2(s-1)-2}{(s-1)^{2}+3} \\
&=2 \frac{s-1}{(s-1)^{2}+3}-\frac{2}{(s-1)^{2}+3} \\
&=2 \frac{s-1}{(s-1)^{2}+3}-\frac{2}{\sqrt{3}} \frac{\sqrt{3}}{(s-1)^{2}+3}
\end{aligned}

Take the inverse Laplace transform of \(Y(s)\) now to recover \(y(t)\).

\begin{aligned}
y(t) &=\mathcal{L}^{-1}\{Y(s)\} \\
&=\mathcal{L}^{-1}\left\{2 \frac{s-1}{(s-1)^{2}+3}-\frac{2}{\sqrt{3}} \frac{\sqrt{3}}{(s-1)^{2}+3}\right\} \\
&=2 \mathcal{L}^{-1}\left\{\frac{s-1}{(s-1)^{2}+3}\right\}-\frac{2}{\sqrt{3}} \mathcal{L}^{-1}\left\{\frac{\sqrt{3}}{(s-1)^{2}+3}\right\} \\
&=2 e^{t} \cos \sqrt{3} t-\frac{2}{\sqrt{3}} e^{t} \sin \sqrt{3} t
\end{aligned}

\subsection*{Problem 19}
\label{sec:org7d295c4}

\subsubsection*{Statement}
\label{sec:org53b2b43}
    \begin{equation*}
y^{\prime \prime}+y=\left\{\begin{array}{ll}
t, & 0 \leq t<1 \\
2-t, & 1 \leq t<2, \\
0, & 2 \leq t<\infty
\end{array} \quad y(0)=0, \quad y^{\prime}(0)=0\right.
\end{equation*}

\subsubsection*{Solution}
\label{sec:org1724030}
    Let \(f(t)\) represent the piecewise function on the right side.
$$
y^{\prime \prime}+y=f(t)=\left\{\begin{array}{ll}
t, & 0 \leq t<1 \\
2-t, & 1 \leq t<2 \\
0, & 2 \leq t<\infty
\end{array}\right.
$$
Because this ODE is linear, the Laplace transform can be applied to solve it. The Laplace transform of a function \(y(t)\) is defined here as
$$
Y(s)=\mathcal{L}\{y(t)\}=\int_{0}^{\infty} e^{-s t} y(t) d t .
$$
Consequently, the first and second derivatives transform as follows.

\begin{aligned}
\mathcal{L}\left\{\frac{d y}{d t}\right\} &=s Y(s)-y(0) \\
\mathcal{L}\left\{\frac{d^{2} y}{d t^{2}}\right\} &=s^{2} Y(s)-s y(0)-y^{\prime}(0)
\end{aligned}

Apply the Laplace transform to both sides of the ODE.

$$
\mathcal{L}\left\{y^{\prime \prime}+y\right\}=\mathcal{L}\{f(t)\}
$$

Use the fact that the transform is a linear operator.

\begin{array}{c}
\mathcal{L}\left\{y^{\prime \prime}\right\}+\mathcal{L}\{y\}=\mathcal{L}\{f(t)\} \\
{\left[s^{2} Y(s)-s y(0)-y^{\prime}(0)\right]+Y(s)=\int_{0}^{\infty} e^{-s t} f(t) d t}
\end{array}

Plug in the initial conditions, \(y(0)=0\) and \(y^{\prime}(0)=0\), and \(f(t)\).

\begin{aligned}
\left[s^{2} Y(s)\right] &+Y(s)=\int_{0}^{1} e^{-s t}(t) d t+\int_{1}^{2} e^{-s t}(2-t) d t+\int_{2}^{\infty} e^{-s t}(0) d t \\
\left(s^{2}+1\right) Y(s) &=\int_{0}^{1} t e^{-s t} d t+2 \int_{1}^{2} e^{-s t} d t-\int_{1}^{2} t e^{-s t} d t \\
&=\frac{1-(s+1) e^{-s}}{s^{2}}+2 \frac{e^{-s}-e^{-2 s}}{s}-\frac{-e^{-2 s}-2 s e^{-2 s}+(s+1) e^{-s}}{s^{2}} \\
&=\frac{1}{s^{2}}+\frac{e^{-2 s}}{s^{2}}-\frac{2 e^{-s}}{s^{2}}
\end{aligned}

Divide both sides by \(s^{2}+1\).

\begin{aligned}
Y(s) &=\frac{1}{s^{2}\left(s^{2}+1\right)}+\frac{e^{-2 s}}{s^{2}\left(s^{2}+1\right)}-\frac{2 e^{-s}}{s^{2}\left(s^{2}+1\right)} \\
&=\frac{1}{s^{2}}-\frac{1}{s^{2}+1}+\left(\frac{1}{s^{2}}-\frac{1}{s^{2}+1}\right) e^{-2 s}-2\left(\frac{1}{s^{2}}-\frac{1}{s^{2}+1}\right) e^{-s}
\end{aligned}

Take the inverse Laplace transform of \(Y(s)\) now to recover \(y(t)\). Note that \(H(t)\) is the Heaviside function, which is defined to be 1 if \(t>0\) and 0 if \(t<0\).

\begin{aligned}
y(t) &=\mathcal{L}^{-1}\{Y(s)\} \\
&=\mathcal{L}^{-1}\left\{\frac{1}{s^{2}}-\frac{1}{s^{2}+1}+\left(\frac{1}{s^{2}}-\frac{1}{s^{2}+1}\right) e^{-2 s}-2\left(\frac{1}{s^{2}}-\frac{1}{s^{2}+1}\right) e^{-s}\right\} \\
&=\mathcal{L}^{-1}\left\{\frac{1}{s^{2}}-\frac{1}{s^{2}+1}\right\}+\mathcal{L}^{-1}\left\{\left(\frac{1}{s^{2}}-\frac{1}{s^{2}+1}\right) e^{-2 s}\right\}-2 \mathcal{L}^{-1}\left\{\left(\frac{1}{s^{2}}-\frac{1}{s^{2}+1}\right) e^{-s}\right\} \\
&=(t-\sin t)+[(t-2)-\sin (t-2)] H(t-2)-2[(t-1)-\sin (t-1)] H(t-1)
\end{aligned}

\subsection*{Problem 22}
\label{sec:org8f83004}

\subsubsection*{Statement}
\label{sec:org9b4242a}
    \begin{equation}
f(t)=t e^{a t}
\end{equation}

\subsubsection*{Solution}
\label{sec:org796b262}
The Laplace transform of a function \(f(t)\) is defined here as
$$
F(s)=\mathcal{L}\{f(t)\}=\int_{0}^{\infty} e^{-s t} f(t) d t
$$

Substitute the given function and evaluate the integral.

\begin{aligned}
F(s) &=\int_{0}^{\infty} e^{-s t} t e^{a t} d t \\
&=\int_{0}^{\infty}\left(-\frac{\partial}{\partial s} e^{-s t}\right) e^{a t} d t \\
&=-\frac{d}{d s} \int_{0}^{\infty} e^{-s t} e^{a t} d t \\
&=-\frac{d}{d s} \int_{0}^{\infty} e^{(a-s) t} d t \\
&=-\frac{d}{d s}\left[\left.\frac{1}{a-s} e^{(a-s) t}\right|_{0} ^{\infty}\right] \\
&=-\frac{d}{d s}\left(\frac{1}{s-a}\right) \\
&=-\left[-\frac{1}{(s-a)^{2}}\right] \\
&=\frac{1}{(s-a)^{2}}
\end{aligned}

\section*{Section 6.3}
\label{sec:orgb756aa9}

\subsection*{Problem 7}
\label{sec:orgba2494b}

\subsubsection*{Statement}
\label{sec:org33454a8}
   \begin{equation*}
f(t)=\left\{\begin{array}{ll}
1, & 0 \leq t<2 \\
e^{-(t-2)}, & t \geq 2
\end{array}\right.
\end{equation*}

\subsubsection*{Solution}
\label{sec:orgbe64809}
Write \(f(t)\) in terms of the Heaviside function, \(H(t)\), which is defined to be 1 if \(t>0\) and 0 if \(t<0\).

\begin{aligned}
f(t) &=1[H(t)-H(t-2)]+e^{-(t-2)} H(t-2) \\
&=H(t)+\left[e^{-(t-2)}-1\right] H(t-2) \\
&=u_{0}(t)+\left[e^{-(t-2)}-1\right] u_{2}(t)
\end{aligned}

\subsection*{Problem 13}
\label{sec:org891e599}

\subsubsection*{Statement}
\label{sec:org56c2633}
    \begin{equation*}
F(s)=\frac{3 !}{(s-2)^{4}}
\end{equation*}

\subsubsection*{Solution}
\label{sec:org525da65}
    Apply the two transforms,
$$
\mathcal{L}\left\{t^{n}\right\}=\frac{n !}{s^{n+1}} \quad \text { and } \quad \mathcal{L}\left\{e^{c t} f(t)\right\}=F(s-c),
$$
together to solve this problem.

\begin{aligned}
f(t) &=\mathcal{L}^{-1}\{F(s)\} \\
&=\mathcal{L}^{-1}\left\{\frac{3 !}{(s-2)^{4}}\right\} \\
&=t^{3} e^{2 t}
\end{aligned}

\subsection*{Problem 16}
\label{sec:orge4fc172}

\subsubsection*{Statement}
\label{sec:org6e17d24}
    \begin{equation*}
F(s)=\frac{e^{-s}+e^{-2 s}-e^{-3 s}-e^{-4 s}}{s}
\end{equation*}

\subsubsection*{Solution}
\label{sec:orgd35ece9}
    Apply the two transforms,
$$
\mathcal{L}\left\{t^{n}\right\}=\frac{n !}{s^{n+1}} \quad \text { and } \quad \mathcal{L}\{f(t-c) H(t-c)\}=F(s) e^{-c s}
$$
together to solve this problem.

\begin{aligned}
f(t) &=\mathcal{L}^{-1}\{F(s)\} \\
&=\mathcal{L}^{-1}\left\{\frac{e^{-s}+e^{-2 s}-e^{-3 s}-e^{-4 s}}{s}\right\} \\
&=\mathcal{L}^{-1}\left\{\frac{1}{s} e^{-s}\right\}+\mathcal{L}^{-1}\left\{\frac{1}{s} e^{-2 s}\right\}-\mathcal{L}^{-1}\left\{\frac{1}{s} e^{-3 s}\right\}-\mathcal{L}^{-1}\left\{\frac{1}{s} e^{-4 s}\right\} \\
&=(t-1)^{0} H(t-1)+(t-2)^{0} H(t-2)-(t-3)^{0} H(t-3)-(t-4)^{0} H(t-4) \\
&=H(t-1)+H(t-2)-H(t-3)-H(t-4) \\
&=u_{1}(t)+u_{2}(t)-u_{3}(t)-u_{4}(t)
\end{aligned}

\subsection*{Problem 20}
\label{sec:orgdf632e1}

\subsubsection*{Statement}
\label{sec:org04e6331}
    \begin{equation*}
F(s)=\frac{1}{9 s^{2}-12 s+3}
\end{equation*}

\subsubsection*{Solution}
\label{sec:org41a911d}
    Observe that the denominator can be written in terms of \(3 s\).
$$
F(s)=\frac{1}{(3 s)^{2}-4(3 s)+3}
$$
Factor the denominator.
$$
F(s)=\frac{1}{[(3 s)-1][(3 s)-3]}
$$
Partially decompose the fraction.
$$
F(s)=\frac{-\frac{1}{2}}{(3 s)-1}+\frac{\frac{1}{2}}{(3 s)-3}
$$
Apply the two transforms,
$$
\mathcal{L}\left\{e^{a t}\right\}=\frac{1}{s-a} \quad \text { and } \quad F(k s)=\mathcal{L}\left\{\frac{1}{k} f\left(\frac{t}{k}\right)\right\},
$$
together to get \(f(t)\)

\begin{aligned}
f(t) &=\mathcal{L}^{-1}\{F(s)\} \\
&=-\frac{1}{2}\left(\frac{1}{3} e^{t / 3}\right)+\frac{1}{2}\left(\frac{1}{3} e^{3 t / 3}\right) \\
&=-\frac{1}{6} e^{t / 3}+\frac{1}{6} e^{t} \\
&=\frac{1}{6}\left(e^{t}-e^{t / 3}\right)
\end{aligned}

\section*{Section 6.4}
\label{sec:org23d19bf}

\subsection*{Problem 1}
\label{sec:org947316c}

\subsubsection*{Statement}
\label{sec:org5f411db}
    \begin{equation*}
y^{\prime \prime}+y=f(t) ; \quad y(0)=0, \quad y^{\prime}(0)=1 ; \quad f(t)=\left\{\begin{array}{ll}
1, & 0 \leq t<3 \pi \\
0, & 3 \pi \leq t<\infty
\end{array}\right.
\end{equation*}

\subsubsection*{Solution}
\label{sec:orgc4cd403}
    Because the ODE is linear, the Laplace transform can be applied to solve it. The Laplace transform of a function \(y(t)\) is defined here as
$$
Y(s)=\mathcal{L}\{y(t)\}=\int_{0}^{\infty} e^{-s t} y(t) d t
$$
Consequently, the first and second derivatives transform as follows.
Apply the Laplace transform to both sides of the ODE.
$$
\mathcal{L}\left\{y^{\prime \prime}+y\right\}=\mathcal{L}\{f(t)\}
$$
Use the fact that the transform is a linear operator.

\begin{array}{c}
\mathcal{L}\left\{y^{\prime \prime}\right\}+\mathcal{L}\{y\}=\mathcal{L}\{f(t)\} \\
{\left[s^{2} Y(s)-s y(0)-y^{\prime}(0)\right]+Y(s)=\int_{0}^{3 \pi} e^{-s t}(1) d t+\int_{3 \pi}^{\infty} e^{-s t}(0) d t}
\end{array}

Plug in the initial conditions, \(y(0)=0\) and \(y^{\prime}(0)=1\).
$$
\left[s^{2} Y(s)-1\right]+Y(s)=\int_{0}^{3 \pi} e^{-s t} d t
$$

As a result of applying the Laplace transform, the ODE has reduced to an algebraic equation for \(Y\), the transformed solution.

\begin{array}{c}
\left(s^{2}+1\right) Y(s)-1=\left.\left(-\frac{1}{s} e^{-s t}\right)\right|_{0} ^{3 \pi} \\
\left(s^{2}+1\right) Y(s)=\frac{1}{s}-\frac{1}{s} e^{-3 \pi s}+1 \\
Y(s)=\frac{1}{s\left(s^{2}+1\right)}-\frac{1}{s\left(s^{2}+1\right)} e^{-3 \pi s}+\frac{1}{s^{2}+1} \\
=\left(\frac{1}{s}-\frac{s}{s^{2}+1}\right)-\left(\frac{1}{s}-\frac{s}{s^{2}+1}\right) e^{-3 \pi s}+\frac{1}{s^{2}+1}
\end{array}

Take the inverse Laplace transform of \(Y(s)\) now to get \(y(t)\).

\begin{aligned}
y(t) &=\mathcal{L}^{-1}\{Y(s)\} \\
&=\mathcal{L}^{-1}\left\{\left(\frac{1}{s}-\frac{s}{s^{2}+1}\right)-\left(\frac{1}{s}-\frac{s}{s^{2}+1}\right) e^{-3 \pi s}+\frac{1}{s^{2}+1}\right\} \\
&=(1-\cos t)-[1-\cos (t-3 \pi)] H(t-3 \pi)+\sin t \\
&=1+\sin t-\cos t-[1-\cos (t-\pi)] H(t-3 \pi) \\
&=1+\sin t-\cos t-(1+\cos t) H(t-3 \pi) \\
&=1+\sin t-\cos t-(1+\cos t) u_{3 \pi}(t)
\end{aligned}



\subsection*{Problem 2}
\label{sec:orgbd8c5be}

\subsubsection*{Solution}
\label{sec:org1b6b00a}

Evaluate the inverse Laplace transforms.

In order to write \(Y(s)\) in terms of known transforms, use partial fraction decomposition.
$$
\frac{1}{s\left(s^{2}+2 s+2\right)}=\frac{A}{s}+\frac{B s+C}{s^{2}+2 s+2}
$$
Multiply both sides by \(s\left(s^{2}+2 s+2\right)\).
$$
1=A\left(s^{2}+2 s+2\right)+(B s+C) s
$$
Plug in three random values of \(s\) to get a system of three equations for \(A, B\), and \$C .\$

\begin{array}{ll}
s=0: & 1=2 A \\
s=1: & 1=5 A+B+C \\
s=2: & 1=10 A+4 B+2 C
\end{array}

Solving this system yields \(A=1 / 2, B=-1 / 2\), and \(C=-1\).
$$
Y(s)=\left(\frac{1 / 2}{s}+\frac{-\frac{1}{2} s-1}{s^{2}+2 s+2}\right) e^{-\pi s}-\left(\frac{1 / 2}{s}+\frac{-\frac{1}{2} s-1}{s^{2}+2 s+2}\right) e^{-2 \pi s}+\frac{1}{s^{2}+2 s+2}
$$
Complete the square in the denominators.

\begin{aligned}
Y(s) &=\left(\frac{1 / 2}{s}+\frac{-\frac{1}{2} s-1}{s^{2}+2 s+1+2-1}\right) e^{-\pi s}-\left(\frac{1 / 2}{s}+\frac{-\frac{1}{2} s-1}{s^{2}+2 s+1+2-1}\right) e^{-2 \pi s}+\frac{1}{s^{2}+2 s+1+2-1} \\
&=\left[\frac{1 / 2}{s}+\frac{-\frac{1}{2} s-1}{(s+1)^{2}+1}\right] e^{-\pi s}-\left[\frac{1 / 2}{s}+\frac{-\frac{1}{2} s-1}{(s+1)^{2}+1}\right] e^{-2 \pi s}+\frac{1}{(s+1)^{2}+1}
\end{aligned}

Make it so that \(s+1\) appears in the numerators.

\begin{aligned}
Y(s)=&\left[\frac{1 / 2}{s}+\frac{-\frac{1}{2}(s+1)-\frac{1}{2}}{(s+1)^{2}+1}\right] e^{-\pi s}-\left[\frac{1 / 2}{s}+\frac{-\frac{1}{2}(s+1)-\frac{1}{2}}{(s+1)^{2}+1}\right] e^{-2 \pi s}+\frac{1}{(s+1)^{2}+1} \\
=&\left[\frac{1 / 2}{s}-\frac{1}{2} \frac{s+1}{(s+1)^{2}+1}-\frac{1}{2} \frac{1}{(s+1)^{2}+1}\right] e^{-\pi s} \\
&-\left[\frac{1 / 2}{s}-\frac{1}{2} \frac{s+1}{(s+1)^{2}+1}-\frac{1}{2} \frac{1}{(s+1)^{2}+1}\right] e^{-2 \pi s}+\frac{1}{(s+1)^{2}+1}
\end{aligned}

Take the inverse Laplace transform of \(Y(s)\) now to get \(y(t)\).
$$
y(t)=\mathcal{L}^{-1}\{Y(s)\}
$$


\begin{aligned}
=\mathcal{L}^{-1}\left\{\left[\frac{1 / 2}{s}\right.\right.&\left.-\frac{1}{2} \frac{s+1}{(s+1)^{2}+1}-\frac{1}{2} \frac{1}{(s+1)^{2}+1}\right] e^{-\pi s} \\
&\left.-\left[\frac{1 / 2}{s}-\frac{1}{2} \frac{s+1}{(s+1)^{2}+1}-\frac{1}{2} \frac{1}{(s+1)^{2}+1}\right] e^{-2 \pi s}+\frac{1}{(s+1)^{2}+1}\right\} \\
=\mathcal{L}^{-1}\left\{\left[\frac{1 / 2}{s}-\right.\right.&\left.\left.\frac{1}{2} \frac{s+1}{(s+1)^{2}+1}-\frac{1}{2} \frac{1}{(s+1)^{2}+1}\right] e^{-\pi s}\right\} \\
&-\mathcal{L}^{-1}\left\{\left[\frac{1 / 2}{s}-\frac{1}{2} \frac{s+1}{(s+1)^{2}+1}-\frac{1}{2} \frac{1}{(s+1)^{2}+1}\right] e^{-2 \pi s}\right\}+\mathcal{L}^{-1}\left\{\frac{1}{(s+1)^{2}+1}\right\}
\end{aligned}

\begin{aligned}
y(t)=&\left[\frac{1}{2}-\frac{1}{2} e^{-(t-\pi)} \cos (t-\pi)-\frac{1}{2} e^{-(t-\pi)} \sin (t-\pi)\right] H(t-\pi) \\
-\left[\frac{1}{2}-\frac{1}{2} e^{-(t-2 \pi)} \cos (t-2 \pi)-\frac{1}{2} e^{-(t-2 \pi)} \sin (t-2 \pi)\right] H(t-2 \pi)+e^{-t} \sin t \\
=&\left(\frac{1}{2}+\frac{1}{2} e^{\pi-t} \cos t+\frac{1}{2} e^{\pi-t} \sin t\right) H(t-\pi) \\
\quad-\left(\frac{1}{2}-\frac{1}{2} e^{2 \pi-t} \cos t-\frac{1}{2} e^{2 \pi-t} \sin t\right) H(t-2 \pi)+e^{-t} \sin t \\
=\frac{1}{2}\left(1+e^{\pi-t} \cos t+e^{\pi-t} \sin t\right) H(t-\pi) \\
-\frac{1}{2}\left(1-e^{2 \pi-t} \cos t-e^{2 \pi-t} \sin t\right) H(t-2 \pi)+e^{-t} \sin t
\end{aligned}

Therefore,
$$
y(t)=\frac{1}{2}\left(1+e^{\pi-t} \cos t+e^{\pi-t} \sin t\right) u_{\pi}(t)-\frac{1}{2}\left(1-e^{2 \pi-t} \cos t-e^{2 \pi-t} \sin t\right) u_{2 \pi}(t)+e^{-t} \sin t
$$  
\end{document}